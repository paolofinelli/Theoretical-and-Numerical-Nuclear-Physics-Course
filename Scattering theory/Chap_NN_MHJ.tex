%%
%% Automatically generated file from DocOnce source
%% (https://github.com/hplgit/doconce/)
%%

% #define PREAMBLE

% #ifdef PREAMBLE
%-------------------- begin preamble ----------------------

% Style: T2 (Springer)
% Use svmono.cls with doconce modifications for bibliography
\documentclass[graybox,sectrefs,envcountresetchap,open=right]{svmonodo}

% Use t2.sty with doconce modifications
\usepackage{t2do}
\special{papersize=193mm,260mm}

\listfiles               % print all files needed to compile this document

\usepackage{relsize,makeidx,color,setspace,amsmath,amsfonts}
\usepackage[table]{xcolor}
\usepackage{bm,microtype}
\usepackage{simplewick}
\usepackage{graphicx}
%\usepackage{minted}
%\usemintedstyle{default}
%\usepackage[framemethod=TikZ]{mdframed}
%myMOD
\let\counterwithout\relax
\let\counterwithin\relax
\usepackage{chngcntr}
\usepackage{textcomp}
%

% Packages for typesetting blocks of computer code
\usepackage{fancyvrb,framed,moreverb}

% Define colors
\definecolor{orange}{cmyk}{0,0.4,0.8,0.2}
\definecolor{darkorange}{rgb}{.71,0.21,0.01}
\definecolor{darkgreen}{rgb}{.12,.54,.11}
\definecolor{myteal}{rgb}{.26, .44, .56}
\definecolor{gray}{gray}{0.45}
\definecolor{mediumgray}{gray}{.8}
\definecolor{lightgray}{gray}{.95}

\colorlet{comment_green}{green!50!black}
\colorlet{string_red}{red!60!black}
\colorlet{keyword_pink}{magenta!70!black}
\colorlet{indendifier_green}{green!70!white}

% New ansi colors
\definecolor{brown}{rgb}{0.54,0.27,0.07}
\definecolor{purple}{rgb}{0.5,0.0,0.5}
\definecolor{darkgray}{gray}{0.25}
\definecolor{darkblue}{rgb}{0,0.08,0.45}
\definecolor{lightred}{rgb}{1.0,0.39,0.28}
\definecolor{lightgreen}{rgb}{0.48,0.99,0.0}
\definecolor{lightblue}{rgb}{0.53,0.81,0.92}
\definecolor{lightpurple}{rgb}{0.87,0.63,0.87}
\definecolor{lightcyan}{rgb}{0.5,1.0,0.83}

% Backgrounds for code
\definecolor{cbg_gray}{rgb}{.95, .95, .95}
\definecolor{bar_gray}{rgb}{.92, .92, .92}

\definecolor{cbg_yellowgray}{rgb}{.95, .95, .85}
\definecolor{bar_yellowgray}{rgb}{.95, .95, .65}

\colorlet{cbg_yellow2}{yellow!10}
\colorlet{bar_yellow2}{yellow!20}

\definecolor{cbg_yellow1}{rgb}{.98, .98, 0.8}
\definecolor{bar_yellow1}{rgb}{.98, .98, 0.4}

\definecolor{cbg_red1}{rgb}{1, 0.85, 0.85}
\definecolor{bar_red1}{rgb}{1, 0.75, 0.85}

\definecolor{cbg_blue1}{rgb}{0.87843, 0.95686, 1.0}
\definecolor{bar_blue1}{rgb}{0.7,     0.95686, 1}
\definecolor{eminence}{RGB}{108,48,130}
\definecolor{commentgreen}{RGB}{2,112,10}
\usepackage{listingsutf8}

% Common lstlisting parameters
\lstset{
  basicstyle=\small \ttfamily,
  breaklines=false,          % break/wrap lines
  breakatwhitespace=true,    % let linebreaks happen at whitespace
  breakindent=40pt,
  tab=,
  tabsize=4,                 % tab means 4 spaces
  %belowskip=\smallskipamount,  % space between code and text below
  xleftmargin=5pt,           % indentation of code frame
  xrightmargin=5pt,
  framexleftmargin=5pt,      % add frame space to the left of code
  %numbers=left,             % put line numbers on the left
  %stepnumber=2,             % stepnumber=1 numbers each line, =n every n lines
  %framerule=0.4pt           % thickness of frame
  aboveskip=1ex,
  showstringspaces=false,    % show spaces in strings with a particular underscore
  showspaces=false,          % show spaces with a particular underscore
  showtabs=false,
  keepspaces=true,
  columns=fullflexible,      % tighter character kerning, like verb
  escapeinside={||},         % for |\pause| in slides and math in code blocks
  extendedchars=\true,       % allows non-ascii chars, does not work with utf-8
   commentstyle=\color{commentgreen},
    keywordstyle=\color{eminence},
    stringstyle=\color{red},
    basicstyle=\small\ttfamily, % basic font setting
    emph={int,char,double,float,unsigned,void,bool},
    emphstyle={\color{blue}},
    escapechar=\&, 
}

% Styles for lstlisting

% Use this one without additional background color
\lstdefinestyle{blue1}{              % blue1 background for code snippets
backgroundcolor=\color{cbg_blue1},
%keywordstyle=\color{blue}\bfseries,
%commentstyle=\color{comment_green}\slshape,
%stringstyle=\color{string_red},
%identifierstyle=\color{darkorange},
%columns=fullflexible,  % tighter character kerning, like verb
}

% Use this one without additional background color
% (same as blue1, but with bar_blue1 frame)
\lstdefinestyle{blue1bar}{           % blue1 background for complete programs
backgroundcolor=\color{cbg_blue1},
frame=trbl,                          % top+right+bottom+left (tb draws double lines at top + bottom)
rulecolor=\color{bar_blue1},         % frame color
%keywordstyle=\color{blue}fseries,
%commentstyle=\color{comment_green}\slshape,
%stringstyle=\color{string_red},
%identifierstyle=\color{darkorange},
%columns=fullflexible,  % tighter character kerning, like verb
}

% Use this one without additional background color
\lstdefinestyle{gray}{
backgroundcolor=\color{cbg_gray},
%frame=trbl,                           % top+right+bottom+left (tb draws double lines at top + bottom)
%framerule=0.4pt                      % thickness of frame
rulecolor=\color{black!40},           % frame color
}

\lstdefinestyle{simple}{
commentstyle={},
}

% end of custom lstdefinestyles

\usepackage[T1]{fontenc}
%\usepackage[latin1]{inputenc}
\usepackage{ucs}
\usepackage[utf8x]{inputenc}

\usepackage{lmodern}         % Latin Modern fonts derived from Computer Modern

% Hyperlinks in PDF:
\definecolor{linkcolor}{rgb}{0,0,0.4}
\usepackage{hyperref}
\hypersetup{
    breaklinks=true,
    colorlinks=true,
    linkcolor=black,
    urlcolor=black,
    citecolor=black,
    filecolor=black,
    %filecolor=blue,
    pdfmenubar=true,
    pdftoolbar=true,
    bookmarksdepth=3   % Uncomment (and tweak) for PDF bookmarks with more levels than the TOC
    }
%\hyperbaseurl{}   % hyperlinks are relative to this root

\setcounter{tocdepth}{2}  % number chapter, section, subsection

% Tricks for having figures close to where they are defined:
% 1. define less restrictive rules for where to put figures
\setcounter{topnumber}{2}
\setcounter{bottomnumber}{2}
\setcounter{totalnumber}{4}
\renewcommand{\topfraction}{0.85}
\renewcommand{\bottomfraction}{0.85}
\renewcommand{\textfraction}{0.15}
\renewcommand{\floatpagefraction}{0.7}
% 2. ensure all figures are flushed before next section
\usepackage[section]{placeins}
% 3. enable begin{figure}[H] (often leads to ugly pagebreaks)
%\usepackage{float}\restylefloat{figure}

% prevent orhpans and widows
\clubpenalty = 10000
\widowpenalty = 10000

\newenvironment{doconceexercise}{}{}
\newcounter{doconceexercisecounter}
% --- begin definition of \listofexercises command ---
\makeatletter
\newcommand\listofexercises{
\chapter*{List of Exercises
          \@mkboth{List of Exercises}{List of Exercises}}
\markboth{List of Exercises}{List of Exercises}
\@starttoc{loe}
}
\newcommand*{\l@doconceexercise}{\@dottedtocline{0}{0pt}{6.5em}}
\makeatother
% --- end definition of \listofexercises command ---


% Let exercises, problems, and projects be numbered per chapter:
\usepackage{chngcntr}
%\counterwithin{doconceexercisecounter}{chapter}

% \subex{} is defined in t2do.sty

% --- end of standard preamble for documents ---


% insert custom LaTeX commands...

\raggedbottom
\makeindex

\usepackage{fancyhdr}
 
\pagestyle{fancy}
\fancyhf{}
%\fancyhead[LE,RO]{Overleaf}
%\fancyhead[RE,LO]{Guides and tutorials}
\fancyfoot[CO]{\textcopyright~Released under CC Attribution-NonCommercial 4.0 license}
%\fancyfoot[LE,RO]{\thepage}

%-------------------- end preamble ----------------------

\begin{document}

% endif for #ifdef PREAMBLE
% #endif

\input{newcommands_keep}

% ------------------- main content ----------------------

% Note on the Springer T2 style: here we use the modifications
% introduced in t2do.sty and svmonodo.sty (both are bundled with DocOnce).


\frontmatter
\setcounter{page}{3}
\pagestyle{headings}


% ----------------- title -------------------------

\thispagestyle{empty}
\hbox{\ \ }
\vfill
\begin{center}
{\huge{\bfseries{
\begin{spacing}{1.25}
{\rule{\linewidth}{0.5mm}} \\[0.4cm]
{Nuclear Many-body Physics \quad \quad - a Computational Approach - Nuclear forces}
\\[0.4cm] {\rule{\linewidth}{0.5mm}} \\[1.5cm]
\end{spacing}
}}}

% ----------------- author(s) -------------------------

\vspace{2.5cm}

{\Large\textsf{Morten Hjorth-Jensen, Michigan State University and University of Oslo, Norway${}^{}$}}\\~\\{\tt https://github.com/mhjensen}\\
{\tt morten.hjorth-jensen@fys.uio.no}\\ {\tt hjensen@msu.edu}\\ [3mm]

\ \\ [2mm]
\vspace{1.5cm}
\begin{flushleft}
Minor editing and modifications by Paolo Finelli ({\tt paolo.finelli@unibo.it} - UniBo).\\
Original materials can be found in:
\begin{itemize}
\item {\tt https://github.com/NuclearStructure/PHY981}
\item {\it  Computational Nuclear Physics-Bridging the scales, from quarks to neutron stars}, Lectures Notes in Physics by Springer, 936 (2017)
\end{itemize}
\vspace{.5cm} 
\textcopyright~Released under CC Attribution-NonCommercial 4.0 license
\end{flushleft}

% ----------------- end author(s) -------------------------


% --- begin date ---
%\ \\ [10mm]
%{\large\textsf{May 8, 2015}}

\end{center}
% --- end date ---
\vfill
\clearpage

\tableofcontents
\clearemptydoublepage
\listofexercises
\clearemptydoublepage



\vspace{1cm} % after toc

\mymainmatter


\chapter{Nuclear forces}
\label{ch:forces}

\noindent
\section{Single-particle and two-particle quantum numbers}
In order to understand the basics of the nucleon-nucleon interaction, we need to define the relevant quantum numbers and how we build up a single-particle state and a two-body state. 

\begin{itemize}
\item For the single-particle states, due to the fact that we have the spin-orbit force, the quantum numbers for the projection of orbital momentum $l$, that is $m_l$, and for spin $s$, that is $m_s$, are no longer so-called good quantum numbers. The total angular momentum $j$ and its projection $m_j$ are then  so-called \emph{good quantum numbers}.

\item This means that the operator $\hat{J}^2$ does not commute with $\hat{L}_z$  or $\hat{S}_z$.  

\item We also start normally with single-particle state functions defined using say the harmonic oscillator. For these functions, we have no explicit dependence on $j$. How can we introduce single-particle wave functions which have $j$ and its projection $m_j$ as quantum numbers? 
\end{itemize}

\noindent
We have that the operators for the orbital momentum are given by
\[
L_x=-i\hbar(y\frac{\partial }{\partial z}-z\frac{\partial }{\partial y})=
yp_z-zp_y,
\]
\[
L_y=-i\hbar(z\frac{\partial }{\partial x}-x\frac{\partial }{\partial z})= zp_x-xp_z,
\]
\[
L_z=-i\hbar(x\frac{\partial }{\partial y}-y\frac{\partial }{\partial x})=xp_y-yp_x.
\]
Since we have a spin orbit force which is strong, it is easy to show that 
the total angular momentum operator
\[
   \hat{J}=\hat{L}+\hat{S}
\]
does not commute with $\hat{L}_z$ and $\hat{S}_z$. To see this, we calculate for example
\begin{eqnarray} 
   [\hat{L}_z,\hat{J}^2]&=&[\hat{L}_z,(\hat{L}+\hat{S})^2] \\ \nonumber
   &=&[\hat{L}_z,\hat{L}^2+\hat{S}^2+2\hat{L}\hat{S}]\\ \nonumber 
   &=& [\hat{L}_z,\hat{L}\hat{S}]=[\hat{L}_z,\hat{L}_x\hat{S}_x+\hat{L}_y\hat{S}_y+\hat{L}_z\hat{S}_z]\ne 0, 
\end{eqnarray}
since we have that $[\hat{L}_z,\hat{L}_x]=i\hbar\hat{L}_y$ and $[\hat{L}_z,\hat{L}_y]=i\hbar\hat{L}_x$. 
We have also
\[
   |\hat{J}|=\hbar\sqrt{J(J+1)},
\]
with the the following degeneracy
\[
   M_J=-J, -J+1, \dots, J-1, J.
\]
With a given value of  $L$ and $S$ we can then determine the possible values of 
 $J$ by studying the $z$ component of  $\hat{J}$. 
It is given by
\[
\hat{J}_z=\hat{L}_z+\hat{S}_z.
\]
The operators $\hat{L}_z$ and $\hat{S}_z$ have the quantum numbers
$L_z=M_L\hbar$ and $S_z=M_S\hbar$, respectively, meaning that
\[
   M_J\hbar=M_L\hbar +M_S\hbar,
\]
or
\[
   M_J=M_L +M_S.
\]
Since the max value of  $M_L$ is $L$ and for  $M_S$ is $S$
we obtain
\[
   (M_J)_{\mathrm{maks}}=L+S.
\]
For nucleons we have that the maximum value of $M_S=m_s=1/2$, yielding
\[
   (m_j)_{\mathrm{max}}=l+\frac{1}{2}.
\]
Using this and the fact that the maximum value of  $M_J=m_j$ is $j$ we have
\[
   j=l+\frac{1}{2}, l-\frac{1}{2}, l-\frac{3}{2}, l-\frac{5}{2}, \dots 
\]
To decide where this series terminates, we use the vector inequality
\[
   |\hat{L}+\hat{S}| \ge \left| |\hat{L}|-|\hat{S}|\right|.
\]
Using $\hat{J}=\hat{L}+\hat{S}$ we get 
\[
   |\hat{J}| \ge |\hat{L}|-|\hat{S}|,
\]
or
\[
   |\hat{J}|=\hbar\sqrt{J(J+1)}\ge |\hbar\sqrt{L(L+1)}-
   \hbar\sqrt{S(S+1)}|.
\]
If we limit ourselves to nucleons only with $s=1/2$ we find that
\[
   |\hat{J}|=\hbar\sqrt{j(j+1)}\ge |\hbar\sqrt{l(l+1)}-
   \hbar\sqrt{\frac{1}{2}(\frac{1}{2}+1)}|.
\]
It is then easy to show that for nucleons there are only two possible values of
$j$ which satisfy the inequality, namely
\[
   j=l+\frac{1}{2}\hspace{0.1cm} \mathrm{or} \hspace{0.1cm}j=l-\frac{1}{2},
\]
and with $l=0$ we get 
\[
   j=\frac{1}{2}.
\]

Let us study some selected examples. We need also to keep in mind that parity is conserved.
The strong and electromagnetic Hamiltonians conserve parity. Thus the eigenstates can be
broken down into two classes of states labeled by their parity $\pi= +1$ or $\pi=-1$.
The nuclear interactions do not mix states with different parity.

For nuclear structure the total parity originates
from the intrinsic parity of the nucleon which is $\pi_{\mathrm{intrinsic}}=+1$ 
and the parities associated with
the orbital angular momenta $\pi_l=(-1)^l$ . The total parity is the product over all nucleons
$\pi = \prod_i \pi_{\mathrm{intrinsic}}(i)\pi_l(i) = \prod_i (-1)^{l_i}$

The basis states we deal with are constructed so that they conserve parity and have thus a definite parity. 

Consider now the single-particle orbits of the $1s0d$ shell. 
For a $0d$ state we have the quantum numbers $l=2$, $m_l=-2,-1,0,1,2$, $s+1/2$, $m_s=\pm 1/2$,
$n=0$ (the number of nodes of the wave function).   This means that we have positive parity and
\[
j=\frac{3}{2}=l-s\hspace{1cm} m_j=-\frac{3}{2},-\frac{1}{2},\frac{1}{2},\frac{3}{2}.
\]
and
\[
j=\frac{5}{2}=l+s\hspace{1cm} m_j=-\frac{5}{2},-\frac{3}{2},-\frac{1}{2},\frac{1}{2},\frac{3}{2},\frac{5}{2}.
\]
Our single-particle wave functions, if we use the harmonic oscillator, do however not contain the quantum numbers $j$ and $m_j$.
Normally what we have is an eigenfunction for the one-body problem defined as
\[
\phi_{nlm_lsm_s}(r,\theta,\phi)=R_{nl}(r)Y_{lm_l}(\theta,\phi)\xi_{sm_s},
\]
where we have used spherical coordinates (with a spherically symmetric potential) and the spherical harmonics
\[
    Y_{lm_l}(\theta,\phi)=P(\theta)F(\phi)=\sqrt{\frac{(2l+1)(l-m_l)!}{4\pi (l+m_l)!}}
                      P_l^{m_l}(cos(\theta))\exp{(im_l\phi)},
\]
with $P_l^{m_l}$ being the so-called associated Legendre polynomials. Examples are
\[
   Y_{00}=\sqrt{\frac{1}{4\pi}},
\]
for $l=m_l=0$, 
\[
   Y_{10}=\sqrt{\frac{3}{4\pi}}cos(\theta),
\]
for $l=1$ and $m_l=0$, 
\[
   Y_{1\pm 1}=\sqrt{\frac{3}{8\pi}}sin(\theta)exp(\pm i\phi),
\]
for  $l=1$ and $m_l=\pm 1$, 
\[
   Y_{20}=\sqrt{\frac{5}{16\pi}}(3cos^2(\theta)-1)
\]
for $l=2$ and $m_l=0$ etc. 

\noindent
How can we get a function in terms of $j$ and $m_j$?
Define now
\[
\phi_{nlm_lsm_s}(r,\theta,\phi)=R_{nl}(r)Y_{lm_l}(\theta,\phi)\xi_{sm_s},
\]
and
\[
\psi_{njm_j;lm_lsm_s}(r,\theta,\phi),
\]
as the state with quantum numbers $jm_j$.
Operating with 
\[
   \hat{j}^2=(\hat{l}+\hat{s})^2=\hat{l}^2+\hat{s}^2+2\hat{l}_z\hat{s}_z+\hat{l}_+\hat{s}_{-}+\hat{l}_{-}\hat{s}_{+},
\]
on the latter state we will obtain admixtures from possible $\phi_{nlm_lsm_s}(r,\theta,\phi)$ states.

\noindent
To see this, we consider the following example and fix
\[
j=\frac{3}{2}=l-s\hspace{1cm} m_j=\frac{3}{2}.
\]
and
\[
j=\frac{5}{2}=l+s\hspace{1cm} m_j=\frac{3}{2}.
\]
It means we can have, with $l=2$ and $s=1/2$ being fixed, in order to have $m_j=3/2$ either $m_l=1$ and $m_s=1/2$ or
$m_l=2$ and $m_s=-1/2$. The two states    
\[
\psi_{n=0j=5/2m_j=3/2;l=2s=1/2}
\]
and
\[
\psi_{n=0j=3/2m_j=3/2;l=2s=1/2}
\]
will have admixtures from $\phi_{n=0l=2m_l=2s=1/2m_s=-1/2}$ and $\phi_{n=0l=2m_l=1s=1/2m_s=1/2}$. 
How do we find these admixtures? Note that we don't specify the values of $m_l$ and $m_s$ 
in the functions $\psi$ since    
$\hat{j}^2$ does not commute with $\hat{L}_z$ and $\hat{S}_z$. 
We operate with 
\[
   \hat{j}^2=(\hat{l}+\hat{s})^2=\hat{l}^2+\hat{s}^2+2\hat{l}_z\hat{s}_z+\hat{l}_+\hat{s}_{-}+\hat{l}_{-}\hat{s}_{+}
\]
on the two $jm_j$ states, that is
\[
\hat{j}^2\psi_{n=0j=5/2m_j=3/2;l=2s=1/2}= \alpha\hbar^2[l(l+1)+\frac{3}{4}+2m_lm_s]\phi_{n=0l=2m_l=2s=1/2m_s=-1/2}+
\]
\[
\beta\hbar^2\sqrt{l(l+1)-m_l(m_l-1)}\phi_{n=0l=2m_l=1s=1/2m_s=1/2},
\]
and
\[
\hat{j}^2\psi_{n=0j=3/2m_j=3/2;l=2s=1/2}= \alpha\hbar^2[l(l+1)+\frac{3}{4}+2m_lm_s]+ \phi_{n=0l=2m_l=1s=1/2m_s=1/2}+
\]
\[
\beta\hbar^2\sqrt{l(l+1)-m_l(m_l+1)}\phi_{n=0l=2m_l=2s=1/2m_s=-1/2}.
\]
This means that the eigenvectors $\phi_{n=0l=2m_l=2s=1/2m_s=-1/2}$ etc are not eigenvectors of $\hat{j}^2$. The above problems gives a $2\times2$ matrix that mixes the vectors $\psi_{n=0j=5/2m_j3/2;l=2m_ls=1/2m_s}$ and $\psi_{n=0j=3/2m_j3/2;l=2m_ls=1/2m_s}$ with the states  $\phi_{n=0l=2m_l=2s=1/2m_s=-1/2}$ and
$\phi_{n=0l=2m_l=1s=1/2m_s=1/2}$. The unknown coefficients $\alpha$ and $\beta$ results from eigenvectors of this matrix. That is, inserting all values $m_l,l,m_s,s$ we obtain the matrix 
\[
\left[ \begin{array} {cc} 19/4 & 2 \\ 2 & 31/4 \end{array} \right]\]
whose eigenvectors are the columns of
\[
\left[ \begin{array} {cc} 2/\sqrt{5} &1/\sqrt{5}  \\ 1/\sqrt{5} & -2/\sqrt{5} \end{array}\right]\]  
These numbers define the so-called Clebsch-Gordan coupling coefficients  (the overlaps between the two basis sets). We can thus write
\[
\psi_{njm_j;ls}=\sum_{m_lm_s}\langle lm_lsm_s|jm_j\rangle\phi_{nlm_lsm_s},
\]
where the coefficients $\langle lm_lsm_s|jm_j\rangle$ are the so-called Clebsch-Gordan coeffficients.



\subsection{Clebsch-Gordan coefficients}

The Clebsch-Gordan coeffficients $\langle lm_lsm_s|jm_j\rangle$ have some interesting properties for us, like the following  orthogonality relations
\[
\sum_{m_1m_2}\langle j_1m_1j_2m_2|JM\rangle\langle j_1m_1j_2m_2|J'M'\rangle=\delta_{J,J'}\delta_{M,M'},
\]
\[
\sum_{JM}\langle j_1m_1j_2m_2|JM\rangle\langle j_1m_1'j_2m_2'|JM\rangle=\delta_{m_1,m_1'}\delta_{m_2,m_2'},
\]
\[
\langle j_1m_1j_2m_2|JM\rangle=(-1)^{j_1+j_2-J}\langle j_2m_2j_1m_1|JM\rangle,
\]
and many others. The latter will turn extremely useful when we are going to define two-body states and interactions in a coupled basis.

\subsection{Quantum numbers and the Schroeodinger equation in relative and CM coordinates}

Summing up, for the single-particle case, we have the following eigenfunctions 
\[
\psi_{njm_j;ls}=\sum_{m_lm_s}\langle lm_lsm_s|jm_j\rangle\phi_{nlm_lsm_s},
\]
where the coefficients $\langle lm_lsm_s|jm_j\rangle$ are the so-called Clebsch-Gordan coeffficients.
The relevant quantum numbers are $n$ (related to the principal quantum number and the number of nodes of the wave function) and 
\[
   \hat{j}^2\psi_{njm_j;ls}=\hbar^2j(j+1)\psi_{njm_j;ls},
\]
\[
   \hat{j}_z\psi_{njm_j;ls}=\hbar m_j\psi_{njm_j;ls},
\]
\[
   \hat{l}^2\psi_{njm_j;ls}=\hbar^2l(l+1)\psi_{njm_j;ls},
\]
\[
   \hat{s}^2\psi_{njm_j;ls}=\hbar^2s(s+1)\psi_{njm_j;ls},
\]
but $s_z$ and $l_z$ do not result in good quantum numbers in a basis where we
use the angular momentum $j$.

For a two-body state where we couple two angular momenta $j_1$ and $j_2$ to a final
angular momentum $J$ with projection $M_J$, we can define a similar transformation in terms
of the Clebsch-Gordan coeffficients
\[
\psi_{(j_1j_2)JM_J}=\sum_{m_{j_1}m_{j_2}}\langle j_1m_{j_1}j_2m_{j_2}|JM_J\rangle\psi_{n_1j_1m_{j_1};l_1s_1}\psi_{n_2j_2m_{j_2};l_2s_2}.
\]
We will write these functions in a more compact form hereafter, namely,
\[
|(j_1j_2)JM_J\rangle=\psi_{(j_1j_2)JM_J},
\]
and
\[
|j_im_{j_i}\rangle=\psi_{n_ij_im_{j_i};l_is_i},
\]
where we have skipped the explicit reference to $l$, $s$ and $n$. The spin of a nucleon is always $1/2$ while the value of $l$ can be deduced from the parity of the state.
It is thus normal to label a state with a given total angular momentum as 
$j^{\pi}$, where $\pi=\pm 1$. 
Our two-body state can thus be written as 
\[
|(j_1j_2)JM_J\rangle=\sum_{m_{j_1}m_{j_2}}\langle j_1m_{j_1}j_2m_{j_2}|JM_J\rangle|j_1m_{j_1}\rangle|j_2m_{j_2}\rangle.
\]
Due to the coupling order of the Clebsch-Gordan coefficient it reads as 
$j_1$ coupled to $j_2$ to yield a final angular momentum $J$. If we invert the order of coupling we would have
\[
|(j_2j_1)JM_J\rangle=\sum_{m_{j_1}m_{j_2}}\langle j_2m_{j_2}j_1m_{j_1}|JM_J\rangle|j_1m_{j_1}\rangle|j_2m_{j_2}\rangle,
\]
and due to the symmetry properties of the Clebsch-Gordan coefficient we have
\begin{eqnarray*}
|(j_2j_1)JM_J\rangle & = & (-1)^{j_1+j_2-J}\sum_{m_{j_1}m_{j_2}}\langle j_1m_{j_1}j_2m_{j_2}|JM_J\rangle|j_1m_{j_1}\rangle|j_2m_{j_2}\rangle \\
& = & (-1)^{j_1+j_2-J}|(j_1j_2)JM_J\rangle.
\end{eqnarray*}
We call the basis $|(j_1j_2)JM_J\rangle$ for the \textbf{coupled basis}, or just $j$-coupled basis/scheme. The basis formed by the simple product of single-particle eigenstates 
$|j_1m_{j_1}\rangle|j_2m_{j_2}\rangle$ is called the \textbf{uncoupled-basis}, or just the $m$-scheme basis. 

\noindent
We have thus the coupled basis 
\[
|(j_1j_2)JM_J\rangle=\sum_{m_{j_1}m_{j_2}}\langle j_1m_{j_1}j_2m_{j_2}|JM_J\rangle|j_1m_{j_1}\rangle|j_2m_{j_2}\rangle.
\]
and the uncoupled basis 
\[
|j_1m_{j_1}\rangle|j_2m_{j_2}\rangle.
\]
The latter can easily be generalized to many single-particle states whereas the first 
needs specific coupling coefficients and definitions of coupling orders. 
The $m$-scheme basis is easy to implement numerically and is used in most standard shell-model codes. 
Our coupled basis obeys also the following relations
\[
   \hat{J}^2|(j_1j_2)JM_J\rangle=\hbar^2J(J+1)|(j_1j_2)JM_J\rangle
\]
\[
   \hat{J}_z|(j_1j_2)JM_J\rangle=\hbar M_J|(j_1j_2)JM_J\rangle,
\]

\section{Components of the force and isospin}

 The nuclear forces are almost charge independent. If we assume they are, 
we can introduce a new quantum number which is conserved. For nucleons only, that is a proton and neutron, we can limit ourselves
to two possible values which allow us to distinguish between the two particles. If we assign an isospin value of $\tau=1/2$ for protons
and neutrons (they belong to an isospin doublet, in the same way as we discussed the spin $1/2$ multiplet), we can define 
the neutron to have isospin projection $\tau_z=+1/2$ and a proton to have $\tau_z=-1/2$. These assignements are the standard choices in low-energy nuclear physics.

\subsection{Isospin}

This leads to the introduction of an additional quantum number called isospin.
We can define a single-nucleon
state function in terms of the quantum numbers $n$, $j$, $m_j$, $l$, $s$, $\tau$ and $\tau_z$. Using our definitions in terms of an uncoupled basis, we had 
\[
\psi_{njm_j;ls}=\sum_{m_lm_s}\langle lm_lsm_s|jm_j\rangle\phi_{nlm_lsm_s},
\]
which we can now extend to
\[
\psi_{njm_j;ls}\xi_{\tau\tau_z}=\sum_{m_lm_s}\langle lm_lsm_s|jm_j\rangle\phi_{nlm_lsm_s}\xi_{\tau\tau_z},
\]
with the isospin spinors defined as 
\[
\xi_{\tau=1/2\tau_z=+1/2}=\left(\begin{array}{c} 1  \\ 0\end{array}\right),
\]
and
\[
\xi_{\tau=1/2\tau_z=-1/2}=\left(\begin{array}{c} 0  \\ 1\end{array}\right).
\]
We can then define the proton state function as 
\[
\psi^p(\mathbf{r})  =\psi_{njm_j;ls}(\mathbf{r})\left(\begin{array}{c} 0  \\ 1\end{array}\right), 
\]
and similarly for neutrons as
\[
\psi^n(\mathbf{r})  =\psi_{njm_j;ls}(\mathbf{r})\left(\begin{array}{c} 1  \\ 0\end{array}\right). 
\]
We can in turn define the isospin Pauli matrices (in the same as we define the spin matrices) as
\[
\hat{\tau}_x =\left(\begin{array}{cc} 0 & 1 \\ 1 & 0 \end{array}\right),
\]
\[
\hat{\tau}_y =\left(\begin{array}{cc} 0 & -\imath \\ \imath & 0 \end{array}\right),
\]
and
\[
\hat{\tau}_z =\left(\begin{array}{cc} 1 & 0 \\ 0 & -1 \end{array}\right),
\]
and operating with $\hat{\tau}_z$ on the proton state function we have
\[
\hat{\tau}_z\psi^p(\mathbf{r})=-\frac{1}{2}\psi^p(\mathbf{r}),
\]
and for neutrons we have
\[
\hat{\tau}\psi^n(\mathbf{r})=\frac{1}{2}\psi^n(\mathbf{r}).
\]
We can now define the so-called charge operator as 
\[
\frac{\hat{Q}}{e} = \frac{1}{2}\left(1-\hat{\tau}_z\right)=\begin{Bmatrix} 0 & 0 \\ 0 & 1 \end{Bmatrix},
\]
which results in 
\[
\frac{\hat{Q}}{e}\psi^p(\mathbf{r})=\psi^p(\mathbf{r}),
\]
and
\[
\frac{\hat{Q}}{e}\psi^n(\mathbf{r})=0,
\]
as it should be. 
The total isospin is defined as
\[
\hat{T}=\sum_{i=1}^A\hat{\tau}_i,
\]
and its corresponding isospin projection as
\[
\hat{T}_z=\sum_{i=1}^A\hat{\tau}_{z_i},
\]
with eigenvalues $T(T+1)$ for $\hat{T}$ and $1/2(N-Z)$ for $\hat{T}_z$, where $N$ is the number of neutrons and $Z$ the number of protons. 
If charge is conserved, the Hamiltonian $\hat{H}$ commutes with $\hat{T}_z$ and all members of a given isospin multiplet
(that is the same value of $T$) have the same energy and there is no $T_z$ dependence and we say that $\hat{H}$ is a scalar in isospin space.

\section{Two-body matrix elements}

Till now we have not said anything about the explicit calculation of two-body matrix elements. It is time to amend this deficiency.
We have till now seen the following definitions of a two-body matrix elements. In $m$-scheme
with quantum numbers $p=j_pm_p$ etc we have a two-body state defined as
\[
|(pq)M\rangle  = a^{\dagger}_pa^{\dagger}_q|\Phi_0\rangle,
\]
where $|\Phi_0\rangle$ is a chosen reference state, say for example the Slater determinant which approximates ${}^{16}\mbox{O}$ with the $0s$ and the $0p$ shells being filled, and $M=m_p+m_q$. Recall that we label single-particle states above the Fermi level as $abcd\dots$ and states below the Fermi level for $ijkl\dots$.  
In case of two-particles in the single-particle states $a$ and $b$ outside ${}^{16}\mbox{O}$ as a closed shell core, say ${}^{18}\mbox{O}$, 
we would write the representation of the Slater determinant as
\[
|^{18}\mathrm{O}\rangle =|(ab)M\rangle  = a^{\dagger}_aa^{\dagger}_b|^{16}\mathrm{O}\rangle=|\Phi^{ab}\rangle.
\]
In case of two-particles removed from say ${}^{16}\mbox{O}$, for example two neutrons in the single-particle states $i$ and $j$, we would write this as
\[
|^{14}\mathrm{O}\rangle =|(ij)M\rangle  = a_ja_i|^{16}\mathrm{O}\rangle=|\Phi_{ij}\rangle.
\]
For a one-hole-one-particle state we have
\[
|^{16}\mathrm{O}\rangle_{1p1h} =|(ai)M\rangle  = a_a^{\dagger}a_i|^{16}\mathrm{O}\rangle=|\Phi_{i}^a\rangle,
\]
and finally for a two-particle-two-hole state we 
\[
|^{16}\mathrm{O}\rangle_{2p2h} =|(abij)M\rangle  = a_a^{\dagger}a_b^{\dagger}a_ja_i|^{16}\mathrm{O}\rangle=|\Phi_{ij}^{ab}\rangle.
\]
Let us go back to the case of two-particles in the single-particle states $a$ and $b$ outside ${}^{16}\mbox{O}$ as a closed shell core, say ${}^{18}\mbox{O}$.
The representation of the Slater determinant is 
\[
|^{18}\mathrm{O}\rangle =|(ab)M\rangle  = a^{\dagger}_aa^{\dagger}_b|^{16}\mathrm{O}\rangle=|\Phi^{ab}\rangle.
\]
The anti-symmetrized matrix element is detailed as 
\[
\langle (ab) M | \hat{V} | (cd) M \rangle = \langle (j_am_aj_bm_b)M=m_a+m_b |  \hat{V} | (j_cm_cj_dm_d)M=m_a+m_b \rangle,
\]
and note that anti-symmetrization means 
\[
\langle (ab) M | \hat{V} | (cd) M \rangle =-\langle (ba) M | \hat{V} | (cd) M \rangle =\langle (ba) M | \hat{V} | (dc) M \rangle,
\]
\[
\langle (ab) M | \hat{V} | (cd) M \rangle =-\langle (ab) M | \hat{V} | (dc) M \rangle. 
\]
This matrix element is the expectation value of 
\[
\langle ^{16}\mathrm{O}|a_ba_a\frac{1}{4}\sum_{pqrs}\langle (pq) M | \hat{V} | (rs) M' \rangle a^{\dagger}_pa^{\dagger}_qa_sa_r a^{\dagger}_ca^{\dagger}_c|^{16}\mathrm{O}\rangle.
\]
We have also defined matrix elements in the coupled basis, the so-called $J$-coupled scheme.
In this case the two-body wave function for two neutrons outside ${}^{16}\mbox{O}$ is written as 
\[
|^{18}\mathrm{O}\rangle_J =|(ab)JM\rangle  = \left\{a^{\dagger}_aa^{\dagger}_b\right\}^J_M|^{16}\mathrm{O}\rangle=N_{ab}\sum_{m_am_b}\langle j_am_aj_bm_b|JM\rangle|\Phi^{ab}\rangle, 
\]
with
\[
|\Phi^{ab}\rangle=a^{\dagger}_aa^{\dagger}_b|^{16}\mathrm{O}\rangle.
\]
We have now an explicit coupling order, where the angular momentum $j_a$ is coupled to the angular momentum $j_b$ to yield a final two-body angular momentum $J$. 
The normalization factor (to be derived below) is
\[
N_{ab}=\frac{\sqrt{1+\delta_{ab}\times (-1)^J}}{1+\delta_{ab}}.
\]
The implementation of the Pauli principle looks different in the $J$-scheme compared with the $m$-scheme. In the latter, no two fermions or more can have the same set of quantum numbers. In the $J$-scheme, when we write a state with the shorthand 
\[
|^{18}\mathrm{O}\rangle_J =|(ab)JM\rangle,
\]
we do refer to the angular momenta only. This means that another way of writing the last state is
\[
|^{18}\mathrm{O}\rangle_J =|(j_aj_b)JM\rangle.
\]
We will use this notation throughout when we refer to a two-body state in $J$-scheme. The Kronecker $\delta$ function in the normalization factor 
refers thus to the values of $j_a$ and $j_b$. If two identical particles are in a state with the same $j$-value, then only even values of the total angular momentum apply.

\noindent
Note also that, using the anti-commuting properties of the creation operators, we obtain
\[
N_{ab}\sum_{m_am_b}\langle j_am_aj_bm_b|JM\rangle|\Phi^{ab}\rangle=-N_{ab}\sum_{m_am_b}\langle j_am_aj_bm_b|JM\rangle|\Phi^{ba}\rangle.
\]
Furthermore, using the property of the Clebsch-Gordan coefficient
\[
\langle j_am_aj_bm_b|JM\rangle=(-1)^{j_a+j_b-J}\langle j_bm_bj_am_a|JM\rangle,
\]
which can be used to show that
\[
|(j_bj_a)JM\rangle  = \left\{a^{\dagger}_ba^{\dagger}_a\right\}^J_M|^{16}\mathrm{O}\rangle=N_{ab}\sum_{m_am_b}\langle j_bm_bj_am_a|JM\rangle|\Phi^{ba}\rangle, 
\]
is equal to 
\[
|(j_bj_a)JM\rangle=(-1)^{j_a+j_b-J+1}|(j_aj_b)JM\rangle.
\]
This relation is important since we will need it when using anti-symmetrized matrix elements in $J$-scheme.

The two-body matrix element is a scalar and since it obeys rotational symmetry, it is diagonal in $J$, 
meaning that the corresponding matrix element in $J$-scheme is 
\[
\langle (j_aj_b) JM | \hat{V} | (j_cj_d) JM \rangle = N_{ab}N_{cd}\sum_{m_am_b m_cm_d}\langle j_am_aj_bm_b|JM\rangle
\]
\[\times \langle j_cm_cj_dm_d|JM\rangle\langle (j_am_aj_bm_b)M |  \hat{V} | (j_cm_cj_dm_d)M \rangle,
\]
and note that of the four $m$-values in the above sum, only three are independent due to the constraint $m_a+m_b=M=m_c+m_d$.
Since
\[
|(j_bj_a)JM\rangle=(-1)^{j_a+j_b-J+1}|(j_aj_b)JM\rangle,
\]
the anti-symmetrized matrix elements need now to obey the following relations
\begin{eqnarray*}
\langle (j_aj_b) JM | \hat{V} | (j_cj_d) JM \rangle & = & (-1)^{j_a+j_b-J+1}\langle (j_bj_a) JM | \hat{V} | (j_cj_d) JM \rangle, \\
\langle (j_aj_b) JM | \hat{V} | (j_cj_d) JM \rangle & = & (-1)^{j_c+j_d-J+1}\langle (j_aj_b) JM | \hat{V} | (j_dj_c) JM \rangle,\\
\langle (j_aj_b) JM | \hat{V} | (j_cj_d) JM \rangle & = & (-1)^{j_a+j_b+j_c+j_d}\langle (j_bj_a) JM | \hat{V} | (j_dj_c) JM \rangle \\ & = & \langle (j_bj_a) JM | \hat{V} | (j_dj_c) JM \rangle,
\end{eqnarray*}
where the last relations follows from the fact that $J$ is an integer and $2J$ is always an even number.

Using the orthogonality properties of the Clebsch-Gordan coefficients,
\[
\sum_{m_am_b}\langle j_am_aj_bm_b|JM\rangle\langle j_am_aj_bm_b|J'M'\rangle=\delta_{JJ'}\delta_{MM'},
\]
and
\[
\sum_{JM}\langle j_am_aj_bm_b|JM\rangle\langle j_am_a'j_bm_b'|JM\rangle=\delta_{m_am_a'}\delta_{m_bm_b'},
\]
we can also express the two-body matrix element in $m$-scheme in terms of that in $J$-scheme, that is, if we multiply with 
\[
\sum_{JMJ'M'}\langle j_am_a'j_bm_b'|JM\rangle\langle j_cm_c'j_dm_d'|J'M'\rangle
\]
from left in
\begin{eqnarray*}
\langle (j_a j_b) JM | \hat{V} | (j_c j_d) JM \rangle & = & N_{ab}N_{cd}\sum_{m_a m_b m_c m_d}\langle j_am_aj_bm_b|JM\rangle\langle j_cm_cj_dm_d|JM\rangle \\
& \times & \langle (j_am_aj_bm_b)M|  \hat{V} | (j_cm_cj_dm_d)M\rangle,
\end{eqnarray*}
we obtain
\begin{eqnarray*}
\langle (j_am_aj_bm_b)M |  \hat{V} | (j_cm_cj_dm_d)M\rangle & = & \frac{1}{N_{ab}N_{cd}}\sum_{JM}\langle j_am_aj_bm_b|JM\rangle\langle j_cm_cj_dm_d|JM\rangle \\
& \times & \langle (j_aj_b) JM | \hat{V} | (j_cj_d) JM \rangle.
\end{eqnarray*}

\section{Phenomenology of nuclear forces}

The aim is to give you an overview over central features of the nucleon-nucleon interaction and how it is constructed, with both technical and theoretical approaches. 

\begin{itemize}
\item The existence of the deuteron with $J^{\pi}=1^+$ indicates that the force between protons and neutrons is attractive at least for the $^3S_1$ partial wave. Interference between Coulomb and nuclear scattering for the proton-proton partial wave $^1S_0$ shows that  the NN force is attractive at least for the $^1S_0$ partial wave. 

\item It has a short range and strong intermediate attraction.

\item Spin dependent, scattering lengths for triplet and singlet states are different,

\item Spin-orbit force. Observation of large polarizations of scattered nucleons perpendicular to the plane of scattering.

\item Strongly repulsive core. The $s$-wave phase shift becomes negative at $\approx 250$ MeV implying that the singlet $S$ has a hard core with range $0.4-0.5$ fm. 

\item Charge independence (almost). Two nucleons in a given two-body state always (almost) experience the same force. Modern interactions break charge and isospin symmetry lightly. That means that the pp, neutron-neutron and pn parts of the interaction will be different for the same quantum numbers. 

\item Non-central. There is a tensor force. First indications from the quadrupole moment of the deuteron pointing to an admixture in the ground state of both $l=2$ ($^3D_1$) and $l=0$ ($^3S_1$) orbital momenta.
\end{itemize}

\noindent
Comparison of the binding energies of
${}^2\mbox{H}$ (deuteron), ${}^3\mbox{H}$ (triton), ${}^4\mbox{He}$ (alpha - particle) show that the nuclear force is of finite range ($1-2$ fm) and very strong within that range.

\noindent
For nuclei with $A>4$, the energy saturates: Volume and binding energies of nuclei are proportional to the mass number $A$ (as we saw from exercise 1).

\noindent
Nuclei are also bound. The average distance
between nucleons in nuclei is about $2$ fm which
must roughly correspond to the range of the
attractive part.


\subsection{Charge Dependence}

\begin{itemize}
 \item After correcting for the electromagnetic interaction, the forces between nucleons (pp, nn, or np) in the same state are almost the same.

 \item \emph{Almost the same}: Charge-independence is slightly broken.

 \item Equality between the pp and nn forces: Charge symmetry.

 \item Equality between pp/nn force and np force: Charge independence.

 \item Better notation: Isospin symmetry, invariance under rotations in isospin
\end{itemize}

\noindent
Charge-symmetry breaking (CSB), after electromagnetic effects
have been removed:
\begin{itemize}
\item $a_{pp}=  -17.3 \pm 0.4 \hspace{0.cm} \mathrm{fm}$

\item $a_{nn}=-18.8 \pm 0.5 \hspace{0.cm} \mathrm{fm}$. Note however discrepancy from $nd$ breakup reactions resulting in  $a_{nn}=-18.72 \pm 0.13 \pm 0.65 \hspace{0.cm} \mathrm{fm}$ and $\pi^- + d \rightarrow \gamma + 2n$ reactions giving  $a_{nn}=-18.93 \pm 0.27 \pm 0.3 \hspace{0.cm} \mathrm{fm}$.
\end{itemize}

\noindent
Charge-independence breaking (CIB)
\begin{itemize}
\item $a_{pn}=  -23.74 \pm 0.02 \hspace{0.cm} \mathrm{fm}$ 
\end{itemize}

\noindent
\section{Symmetries of the Nucleon-Nucleon (NN) Force}

\begin{itemize}
\item Translation invariance

\item Galilean invariance

\item Rotation invariance in space

\item Space reflection invariance

\item Time reversal invariance

\item Invariance under the interchange of particle $1$ and $2$

\item Almost isospin symmetry
\end{itemize}

\noindent
Here we display a typical way to parametrize (non-relativistic expression) the nuclear two-body force
in terms of some operators, the central part, the spin-spin part and the central force.
\[
V(\mathbf{r})= \left\{ C_c + C_\mathbf{\sigma} \mathbf{\sigma}_1\cdot\mathbf{\sigma}_2
 + C_T \left( 1 + {3\over m_\alpha r} + {3\over\left(m_\alpha r\right)^2}\right) S_{12} (\hat r)\right. 
\]
\[
\left. + C_{SL} \left( {1\over m_\alpha r} + {1\over \left( m_\alpha r\right)^2}
\right) \mathbf{L}\cdot \mathbf{S}
\right\} \frac{e^{-m_\alpha r}}{m_\alpha r}
\]

\subsection{Relative and CoM system, quantum numbers}

When solving the scattering equation or solving the two-nucleon problem, it is convenient to rewrite the Schroedinger equation, due to
the spherical symmetry of the Hamiltonian, in relative and center-of-mass coordinates. This will also define the quantum numbers of the relative and center-of-mass system and will aid us later in solving
the so-called Lippman-Schwinger equation for the scattering problem. 

\noindent
We define the center-of-mass (CoM)  momentum as
 \[
    \mathbf{K}=\sum_{i=1}^A\mathbf{k}_i,
 \]
with $\hbar=c=1$ the wave number $k_i=p_i$, with $p_i$ the pertinent momentum of a single-particle state. 
We have also the relative momentum
\[
    \mathbf{k}_{ij}=\frac{1}{2}(\mathbf{k}_i-\mathbf{k}_j).
 \]
We will below skip the indices $ij$ and simply write $\mathbf{k}$
In a similar fashion we can define the CoM coordinate
 \[
     \mathbf{R}=\frac{1}{A}\sum_{i=1}^{A}\mathbf{r}_i,
 \]
 and the relative distance 
\[
    \mathbf{r}_{ij}=(\mathbf{r}_i-\mathbf{r}_j).
 \]
With the definitions
 \[
    \mathbf{K}=\sum_{i=1}^A\mathbf{k}_i,
 \]
and
\[
    \mathbf{k}_{ij}=\frac{1}{2}(\mathbf{k}_i-\mathbf{k}_j).
 \]
we can rewrite the two-particle kinetic energy (note that we use $\hbar=c=1$ as 
\[
\frac{\mathbf{k}_1^2}{2m_n}+\frac{\mathbf{k}_2^2}{2m_n}=\frac{\mathbf{k}^2}{m_n}+\frac{\mathbf{K}^2}{4m_n},
\]
where $m_n$ is the average of the proton and the neutron masses. 

\noindent
Since the two-nucleon interaction depends only on the relative distance, this means that we can separate Schroedinger's equation in an equation for the center-of-mass motion and one for the relative motion.

\noindent
With an equation for the relative motion only and a separate one for the center-of-mass motion we need to redefine the two-body quantum numbers.

\noindent
Previously we had a two-body state vector defined as $|(j_1j_2)JM_J\rangle$ in a coupled basis. 
We will now define the quantum numbers for the relative motion. Here we need to define new orbital momenta (since these are the quantum numbers which change). 
We define 
\[
\hat{l}_1+\hat{l}_2=\hat{\lambda}=\hat{l}+\hat{L},
\]
where $\hat{l}$ is the orbital momentum associated with the relative motion and
$\hat{L}$ the corresponding one linked with the CoM. The total spin $S$ is unchanged since it acts in a different space. We have thus that
\[
\hat{J}=\hat{l}+\hat{L}+\hat{S},
\]
which allows us to define the angular momentum of the relative motion
\[
{ \cal J} =  \hat{l}+\hat{S},
\]
where ${ \cal J}$ is the total angular momentum of the relative motion.


The total two-nucleon state function has to be anti-symmetric. The total function contains a spatial part, a spin part and an isospin part. If isospin is conserved, this leads to in case we have an $s$-wave with spin $S=0$ to an isospin 
two-body state with $T=1$ since the spatial part is symmetric and the spin part is anti-symmetric. 

Since the projections for $T$ are $T_z=-1,0,1$, we can have a $pp$, an $nn$ and a $pn$ state.

For $l=0$ and $S=1$, a so-called triplet state, $^3S_1$, we must have $T=0$, meaning that we have only one state, a $pn$ state. For other partial waves, the following table lists states up to $f$ waves.
We can systemize this in a table as follows, recalling that $|\mathbf{l}-\mathbf{S}| \le |\mathbf{J}| \le |\mathbf{l}+\mathbf{S}|$,  

\begin{center}

\begin{tabular}{cccccccc}
\hline
\multicolumn{1}{c}{ $^{2S+1}l_J$ } & \multicolumn{1}{c}{ $J$ } & \multicolumn{1}{c}{ $l$ } & \multicolumn{1}{c}{ $S$ } & \multicolumn{1}{c}{ $T$ } & \multicolumn{1}{c}{ $\vert pp\rangle$ } & \multicolumn{1}{c}{ $\vert pn\rangle$ } & \multicolumn{1}{c}{ $\vert nn\rangle$ } \\
\hline
$^{1}S_0$    & 0   & 0   & 0   & 1   & yes               & yes               & yes               \\
$^{3}S_1$    & 1   & 0   & 1   & 0   & no                & yes               & no                \\
$^{3}P_0$    & 0   & 1   & 1   & 1   & yes               & yes               & yes               \\
$^{1}P_1$    & 1   & 1   & 0   & 0   & no                & yes               & no                \\
$^{3}P_1$    & 1   & 1   & 1   & 1   & yes               & yes               & yes               \\
$^{3}P_2$    & 2   & 1   & 1   & 1   & yes               & yes               & yes               \\
$^{3}D_1$    & 1   & 2   & 1   & 0   & no                & yes               & no                \\
$^{3}F_2$    & 2   & 3   & 1   & 1   & yes               & yes               & yes               \\
\hline
\end{tabular}

\end{center}

\subsection{Components of the force and quantum numbers}

The tensor force is given by
\[
S_{12} (\hat r) = \frac{3}{r^2}\left(\mathbf{\sigma}_1\cdot \mathbf{r}\right) \left(\mathbf{\sigma}_2\cdot \mathbf{r}\right) -\mathbf{\sigma}_1\cdot\mathbf{\sigma}_2\]
where the Pauli matrices are defined as
\[
\sigma_x =\begin{Bmatrix} 0 & 1 \\ 1 & 0 \end{Bmatrix},
\]
\[
\sigma_y =\begin{Bmatrix} 0 & -\imath \\ \imath & 0 \end{Bmatrix},
\]
and
\[
\sigma_z =\begin{Bmatrix} 1 & 0 \\ 0 & -1 \end{Bmatrix},
\]
with the properties $\sigma = 2\mathbf{S}$ (the spin of the system, being $1/2$ for nucleons), 
$\sigma^2_x=\sigma^2_y=\sigma_z=\mathbf{1}$ and
obeying the commutation and anti-commutation relations $\{\sigma_x,\sigma_y\} =0$
$[\sigma_x,\sigma_y] =\imath\sigma_z$ etc.

\noindent
When we look at the expectation value of 
$\langle \mathbf{\sigma}_1\cdot\mathbf{\sigma}_2\rangle$, we can rewrite this expression in terms of the
spin $\mathbf{S}=\mathbf{s}_1+\mathbf{s}_2$, resulting in 
\[
\langle\mathbf{\sigma}_1\cdot\mathbf{\sigma}_2\rangle=2(S^2-s_1^2-s_2^2)=2S(S+1)-3,
\]
where we $s_1=s_2=1/2$ leading to
\[
\left\{ \begin{array}{cc} \langle\mathbf{\sigma}_1\cdot\mathbf{\sigma}_2\rangle=1 &  \mathrm{if} \hspace{0.2cm} S=1\\
\langle\mathbf{\sigma}_1\cdot\mathbf{\sigma}_2\rangle=-3 & \mathrm{if} \hspace{0.2cm} S=0\\\end{array}\right.
\]

\noindent
Similarly, the expectation value of the spin-orbit term is 
\[
\langle \mathbf{l}\mathbf{S} \rangle = \frac{1}{2}\left( J(J+1)-l(l+1)-S(S+1)\right),
\]
which means that for $s$-waves with either $S=0$ and thereby $J=0$ or $S=1$ and $J=1$, 
the expectation value for the
spin-orbit force is zero. With the above phenomenological model, the
only contributions to the expectation value of the potential energy for $s$-waves
stem  from the central and the spin-spin components since the
expectation value of the tensor force is also zero.

\noindent
 For $s=1/2$ spin values only for two nucleons, the expectation value of the tensor force operator is 

\begin{center}

\begin{tabular}{cccc}
\hline
\multicolumn{1}{c}{  } & \multicolumn{1}{c}{ $l'$ } & \multicolumn{1}{c}{  } & \multicolumn{1}{c}{  } \\
\hline
$l$   & $J+1$                         & $J$ & $J-1$                         \\
\hline
      &                               &     &                               \\
$J+1$ & $-\frac{2J(J+2)}{2J+1}$       & 0   & $\frac{6\sqrt{J(J+1)}}{2J+1}$ \\
      &                               &     &                               \\
$J$   & 0                             & 2   & 0                             \\
      &                               &     &                               \\
$J-1$ & $\frac{6\sqrt{J(J+1)}}{2J+1}$ & 0   & $-\frac{2(2J+1)}{2J+1}$       \\
      &                               &     &                               \\
\hline
\end{tabular}

\end{center}


\noindent
We will derive these expressions after we have discussed the Wigner-Eckart theorem. 

\noindent
If we now add isospin to our simple $V_4$ interaction model, we end up with $8$ operators, popularly dubbed $V_8$ interaction model. The explicit form reads
\[
V(\mathbf{r})= \left\{ C_c + C_\mathbf{\sigma} \mathbf{\sigma}_1\cdot\mathbf{\sigma}_2
 + C_T \left( 1 + {3\over m_\alpha r} + {3\over
\left(m_\alpha r\right)^2}\right) S_{12} (\hat r)\right. 
\]
\[
\left. + C_{SL} \left( {1\over m_\alpha r} + {1\over \left( m_\alpha r\right)^2}
\right) \mathbf{L}\cdot \mathbf{S}
\right\} \frac{e^{-m_\alpha r}}{m_\alpha r}
\]
\[
+ \left\{ C_{c\tau} + C_{\sigma\tau}\mathbf{\sigma}_1\cdot\mathbf{\sigma}_2
 + C_{T\tau} \left( 1 + {3\over m_\alpha r} + {3\over
\left(m_\alpha r\right)^2}\right) S_{12} (\hat r)\right. 
\]
\[
\left. + C_{SL\tau} \left( {1\over m_\alpha r} + {1\over \left( m_\alpha r\right)^2}
\right) \mathbf{L}\cdot \mathbf{S}
\right\}\mathbf{\tau}_1\cdot\mathbf{\tau}_2 \frac{e^{-m_\alpha r}}{m_\alpha r}
\]
The total two-nucleon state function has to be anti-symmetric. The total function contains a spatial part, a spin part and an isospin part. If isospin is conserved, this leads to in case we have an $s$-wave with spin $S=0$ to an isospin 
two-body state with $T=1$ since the spatial part is symmetric and the spin part is anti-symmetric. 

\noindent
Since the projections for $T$ are $T_z=-1,0,1$, we can have a $pp$, an $nn$ and a $pn$ state.

\noindent
For $l=0$ and $S=1$, a so-called triplet state, $^3S_1$, we must have $T=0$, meaning that we have only one state, a $pn$ state. For other partial waves, see exercises below. 



\subsection{Phenomenology of one-pion exchange}

The one-pion exchange contribution (see derivation below), can be written as 
\[
V_{\pi}(\mathbf{r})= -\frac{f_{\pi}^{2}}{4\pi m_{\pi}^{2}}\mathbf{ \tau}_1\cdot\mathbf{\tau}_2
\frac{1}{3}\left\{\mathbf{ \sigma}_1\cdot\mathbf{ \sigma}_2+\left( 1 + {3\over m_\pi r} + {3\over\left(m_\pi r\right)^2}\right) S_{12} (\hat r)\right\} \frac{e^{-m_\pi r}}{m_\pi r}.
\]
Here the constant $f_{\pi}^{2}/4\pi\approx 0.08$ and the mass of the pion is $m_\pi\approx 140$ MeV/$\mbox{c}^2$.  

\noindent
Let us look closer at specific partial waves for which one-pion exchange is applicable. If we have $S=0$ and $T=0$, the 
orbital momentum has to be an odd number in order for the total anti-symmetry to be obeyed. For $S=0$, the tensor force component is zero, meaning that 
the only contribution is 
\[
V_{\pi}(\mathbf{r})=\frac{3f_{\pi}^{2}}{4\pi m_{\pi}^{2}}\frac{e^{-m_\pi r}}{m_\pi r},
\]
since $\langle\mathbf{ \sigma}_1\cdot\mathbf{ \sigma}_2\rangle=-3$, that is we obtain a repulsive contribution to partial waves like 
$^1P_0$.

\noindent
Since $S=0$ yields always a zero tensor force contribution, for the combination of $T=1$ and then even $l$ values, we get an attractive contribution
\[
V_{\pi}(\mathbf{r})=-\frac{f_{\pi}^{2}}{4\pi m_{\pi}^{2}}\frac{e^{-m_\pi r}}{m_\pi r}.
\]
With $S=1$ and $T=0$, $l$ can only take even values in order to obey the anti-symmetry requirements and we get
\[
V_{\pi}(\mathbf{r})= -\frac{f_{\pi}^{2}}{4\pi m_{\pi}^{2}}
\left(1+( 1 + {3\over m_\pi r} + {3\over\left(m_\pi r\right))^2}) S_{12} (\hat r)\right) \frac{e^{-m_\pi r}}{m_\pi r},
\]
while for $S=1$ and $T=1$, $l$ can only take odd values, resulting in a repulsive contribution 
\[
V_{\pi}(\mathbf{r})= \frac{1}{3}\frac{f_{\pi}^{2}}{4\pi m_{\pi}^{2}}\left(1+( 1 + {3\over m_\pi r} + {3\over\left(m_\pi r\right)^2}) S_{12} (\hat r)\right) \frac{e^{-m_\pi r}}{m_\pi r}.
\]
The central part of one-pion exchange interaction, arising from the spin-spin term,  
is thus attractive for $s$-waves and all even $l$ values. For $p$-waves and all other odd values
it is repulsive. However, its overall strength is weak. This is discussed further in one of exercises below.

\section{Models for nuclear forces and derivation of non-relativistic expressions}

To describe the interaction between the various baryons and mesons of the previous
table we choose the following phenomenological
lagrangians
for spin $1/2$ baryons
\[
   {\cal L}_{ps} =g^{ps}\overline{\Psi}\gamma^{5}
   \Psi\phi^{(ps)},
\]
\[
   {\cal L}_{s} =g^{s}\overline{\Psi}\Psi\phi^{(s)},
\]
and
\[
   {\cal L}_{v} =g^{v}\overline{\Psi}\gamma_{\mu}\Psi\phi_{\mu}^{(v)}
   +g^{t}\overline{\Psi}\sigma^{\mu\nu}\Psi\left
   (\partial_{\mu}\phi_{\nu}^{(v)}
   -\partial_{\nu}\phi_{\mu}^{(v)}\right),
\]
for pseudoscalar (ps), scalar (s) and vector (v) coupling, respectively.
The factors $g^{v}$ and $g^{t}$ are the vector
and tensor coupling constants, respectively.

For spin $1/2$ baryons, the fields $\Psi$ are expanded
in terms of the Dirac spinors (positive energy
solution shown here with $\overline{u}u=1$)
\[
   u(k\sigma)=\sqrt{\frac{E(k)+m}{2m}}
	  \left(\begin{array}{c} \chi\\ \\
	  \frac{\mathbf{\sigma}\mathbf{k}}{E(k)+m}\chi
	  \end{array}\right), 
\]
with $\chi$ the familiar Pauli spinor and $E(k) =\sqrt{m^2 +|\mathbf{k}|^2}$. 
The positive energy part of the field $\Psi$ reads
\[
\Psi (x)={\displaystyle \frac{1}{(2\pi )^{3/2}}
        \sum_{\mathbf{k}\mathbf{\sigma}}u(k\mathbf{\sigma})\exp{-(ikx)}a_{\mathbf{k}\mathbf{\sigma}}},
\]
with $a$ being a fermion annihilation operator.

\noindent
Expanding the free Dirac spinors
in terms of $1/m$ ($m$ is here the mass of the relevant baryon) 
results, to lowest order, in the familiar non-relativistic
expressions for baryon-baryon potentials.
The configuration space version of the interaction can be approximated as
\[
V(\mathbf{r})= \left\{ C^0_C + C^1_C + C_\sigma 
\mathbf{\sigma}_1\cdot\mathbf{\sigma}_2
 + C_T \left( 1 + {3\over m_\alpha r} + {3\over
\left(m_\alpha r\right)^2}
\right) S_{12} (\hat r)\right.
\]
\[
+ C_{SL}\left. \left( {1\over m_\alpha r} + {1\over \left( m_\alpha r\right)^2}
\right) \mathbf{L}\cdot \mathbf{S}
\right\} \frac{\exp{-(m_\alpha r)}}{m_\alpha r},
\]
where $m_{\alpha}$ is the mass of the relevant meson and
$S_{12}$ is the familiar tensor term.

\noindent
We derive now the non-relativistic one-pion exchange interaction.

\noindent
Here $p_{1}$, $p_{1}'$, $p_{2}$, $p_{2}'$ and $k=p_{1}-p_{1}'$ denote 
four-momenta.  
The vertices are 
given by the pseudovector Lagrangian
\[
{\cal L}_{pv}=\frac{f_{\pi}}{m_{\pi}}\overline{\psi}\gamma_{5}\gamma_{\mu}
\psi\partial^{\mu}\phi_{\pi}.
\]
 From the Feynman diagram rules we can write the two-body interaction as  
\[
V^{pv}=\frac{f_{\pi}^{2}}{m_{\pi}^{2}}\frac{\overline{u}(p_{1}')\gamma_{5}
\gamma_{\mu}(p_{1}-p_{1}')^{\mu}u(p_{1})\overline{u}(p_{2}')\gamma_{5}
\gamma_{\nu}(p_{2}'-p_{2})^{\nu}u(p_{2})}{(p_{1}-p_{1}')^{2}-m_{\pi}^{2}}.
\]
The factors $p_{1}-p_{1}'=p_{2}'-p_{2}$ are both the four-momentum of the 
exchanged meson and come from the derivative of the meson field in 
the interaction Lagrangian. 
The Dirac spinors obey 
\begin{eqnarray*}
\gamma_{\mu}p^{\mu}u(p)&=&mu(p) \nonumber \\
\overline{u}(p)\gamma_{\mu}p^{\mu}&=&m\overline{u}(p). \nonumber
\end{eqnarray*} 

\noindent
Using these relations, together with $\{\gamma_{5},\gamma_{\mu}\}=0$, we find 
\begin{eqnarray*}
\overline{u}(p_{1}')\gamma_{5}\gamma_{\mu}(p_{1}-p_{1}')^{\mu}u(p_{1})
&=&m\overline{u}(p_{1}')\gamma_{5}u(p_{1})+\overline{u}(p_{1}')\gamma_{\mu}
p_{1}'^{\mu}\gamma_{5}u(p_{1}) \nonumber \\
 &=&2m\overline{u}(p_{1}')\gamma_{5}u(p_{1}) \nonumber
\end{eqnarray*}
and 
\[
\overline{u}(p_{2}')\gamma_{5}\gamma_{\mu}(p_{2}'-p_{2})^{\mu}=
-2m\overline{u}(p_{2}')\gamma_{5}u(p_{1}).
\]
We get
\[
V^{pv}=-\frac{f_{\pi}^{2}}{m_{\pi}^{2}}4m^{2}\frac{\overline{u}(p_{1}')
\gamma_{5}u(p_{1})\overline{u}(p_{2}')\gamma_{5}u(p_{2})}{(p_{1}-p_{1}')
^{2}-m_{\pi}^{2}}.
\]
By inserting expressions for the Dirac spinors, we find
\begin{eqnarray*}
\overline{u}(p_{1}')\gamma_{5}u(p_{1})&=&\sqrt{\frac{(E_{1}'+m)(E_{1}+m)}
{4m^{2}}}\left(\begin{array}{cc}\chi^{\dagger}&-\frac{\sigma_{1}\cdot{
\bf p_{1}}}{E_{1}'
+m}\chi^{\dagger}\end{array}\right)\left(\begin{array}{cc}0&1\\1&0\end{array}
\right)\nonumber \\
 &&\times \left(\begin{array}{c}\chi\\ \frac{\sigma_{1}\cdot\mathbf{p_{1}}}{E_{1}+m}\chi
\end{array}\right) 
\nonumber \\
 &=&\sqrt{\frac{(E_{1}'+m)(E_{1}+m)}{4m^{2}}}\left(\frac{\sigma_{1}\cdot
\mathbf{p_{1}}}{E_{1}+m}-\frac{\sigma_{1}\cdot\mathbf{p_{1}'}}{E_{1}'+m}\right) 
\nonumber 
\end{eqnarray*}
Similarly
\[
\overline{u}(p_{2}')\gamma_{5}u(p_{2})=\sqrt{\frac{(E_{2}'+m)(E_{2}+m)}
{4m^{2}}}\left(\frac{\sigma_{2}\cdot \mathbf{p}_{2}}{E_{2}+m}-
\frac{\sigma_{2}\cdot\mathbf{p'}_{2}}{E_{2}'+m}\right).
\]
In the CM system we have $\mathbf{p}_{2}=-\mathbf{p}_{1}$, $\mathbf{p'}_{2}=
-\mathbf{p'}_{1}$ and so $E_{2}=E_{1}$, $E_{2}'=E_{1}'$.  
We can then write down the relativistic contribution 
to the NN potential in the CM system: 
\begin{eqnarray}
V^{pv}&=&-\frac{f_{\pi}^{2}}{m_{\pi}^{2}}4m^{2}\frac{1}{(p_{1}-p_{1}')^{2}-
m_{\pi}^{2}}\frac{(E_{1}+m)(E_{1}'+m)}{4m^{2}} \nonumber \\ 
 &\times&\left(\frac{\sigma_{1}\cdot\mathbf{p}_{1}}{E_{1}+m}-\frac{\sigma_{1}
\cdot\mathbf{p'}_{1}}{E_{1}'+m}\right)\left(\frac{\sigma_{2}\cdot\mathbf{p}_{1}}
{E_{1}+m}-\frac{\sigma_{2}\cdot\mathbf{p'}_{1}}{E_{1}'+m}\right). \nonumber
\end{eqnarray}
In the non-relativistic limit we have to lowest order 
\[
E_{1}=\sqrt{\mathbf{p}_{1}^{2}+m^{2}}\approx m \approx E_{1}'
\]
and then $(p_{1}-p_{1}')^{2}=-\mathbf{k}^{2}$, so we get 
for the contribution to the NN potential
\begin{eqnarray}
V^{pv}&=&-\frac{f_{\pi}^{2}}{m_{\pi}^{2}}4m^{2}\frac{1}{\mathbf{k}^{2}+m^{2}}
\frac{2m\cdot 2m}{4m^{2}}\frac{\sigma_{1}}{2m}\cdot(\mathbf{p}_{1}-\mathbf{p'}_{1})
\frac{\sigma_{2}}{2m}\cdot (\mathbf{p}_{1}-\mathbf{p'}_{1}) \nonumber \\ 
 &=&-\frac{f_{\pi}^{2}}{m_{\pi}^{2}}
\frac{(\sigma_{1}\cdot\mathbf{k})(\sigma_{2}\cdot\mathbf{k})}{\mathbf{k}^{2}+m_{\pi}^{2}}.
\nonumber
\end{eqnarray}
We have omitted exchange terms and the isospin term $\mathbf{\tau}_1\cdot\mathbf{\tau}_2$.
\noindent
We have
\[
V^{pv}(k)=-\frac{f_{\pi}^{2}}{m_{\pi}^{2}}
\frac{(\sigma_{1}\cdot\mathbf{k})(\sigma_{2}\cdot\mathbf{k})}{\mathbf{k}^{2}+m_{\pi}^{2}}.
\]
In coordinate space we have
\[
V^{pv}(r)=\int\frac{d^3k}{(2\pi)^3}e^{i\mathbf{kr}}V^{pv}(k)
\]
resulting in
\[
  V^{pv}(r)=-\frac{f_{\pi}^{2}}{m_{\pi}^{2}}
\sigma_{1}\cdot{\nabla}\sigma_{2}\cdot{\nabla}
\int\frac{d^3k}{(2\pi)^3}e^{i\mathbf{kr}}\frac{1}{\mathbf{k}^{2}+m_{\pi}^{2}}.
\]
We obtain
\[
V^{pv}(r)=-\frac{f_{\pi}^{2}}{m_{\pi}^{2}}\sigma_{1}\cdot{\nabla}\sigma_{2}\cdot{\nabla}\frac{e^{-m_{\pi}r}}{r}.
\]
Carrying out the differentation of
\[
V^{pv}(r)=-\frac{f_{\pi}^{2}}{m_{\pi}^{2}}\sigma_{1}\cdot{\nabla}\sigma_{2}\cdot{\nabla}\frac{e^{-m_{\pi}r}}{r}.
\]
we arrive at the famous one-pion exchange potential with central and tensor parts
\[
V(\mathbf{r})= -\frac{f_{\pi}^{2}}{m_{\pi}^{2}}\left\{C_{\sigma}\mathbf{\sigma}_1\cdot\mathbf{\sigma}_2+ C_T \left( 1 + \frac{3}{m_\alpha r} + \frac{3}{\left(m_\alpha r\right)^2}\right) S_{12}(\hat r)\right\}\frac{\exp{-m_\pi r}}{m_\pi r}.
\]
For the full potential add the exchange part and the $\mathbf{\tau}_1\cdot\mathbf{\tau}_2$ term as well. (Subtle point: there is a divergence which gets cancelled by using cutoffs) This leads to coefficients $C_{\sigma}$ and $C_T$ which are fitted to data.


When we perform similar non-relativistic expansions for scalar and vector mesons we obtain
for the $\sigma$ meson
\[
V^{\sigma}= g_{\sigma NN}^{2}\frac{1}{\mathbf{k}^{2}+m_{\sigma}^{2}}\left (-1+\frac{\mathbf{q}^{2}}{2M_N^2}
-\frac{\mathbf{k}^{2}}{8M_N^2}-\frac{\mathbf{LS}}{2M_N^2}\right).
\]
We note an attractive central force and spin-orbit force. This term has an intermediate range.
We have defined $1/2(p_{1}+p_{1}')=\mathbf{q}$.
For the full potential add the exchange part and the isospin dependence as well.



We obtain
for the $\omega$ meson
\[
V^{\omega}= g_{\omega NN}^{2}\frac{1}{\mathbf{k}^{2}+m_{\omega}^{2}}\left (1-3\frac{\mathbf{LS}}{2M_N^2}\right).
\]
We note a repulsive central force and an attractive spin-orbit force. This term has  short range.
For the full potential add the exchange part and the isospin dependence as well.


Finally 
for the $\rho$ meson
\[
V^{\rho}= g_{\rho NN}^{2}\frac{\mathbf{k}^{2}}{\mathbf{k}^{2}+m_{\rho}^{2}}\left (
-2\sigma_{1}\sigma_{2}+S_{12}(\hat{k})\right)\tau_{1}\tau_{2}.
\]
We note a tensor force with sign opposite to that of the pion. This term has  short range. For the full potential add the exchange part and the isospin dependence as well.

\noindent
Important topics:

\begin{itemize}
%\item Can use a one-boson exchange picture to construct a nucleon-nucleon interaction a la QED

\item Non-relativistic approximation yields amongst other things a spin-orbit force which is much stronger than in atoms.

\item At large intermediate distances pion exchange dominates while  pion resonances (other mesons) dominate at intermediate and short range 

%\begin{itemize}

\item Potentials are parameterized to fit selected two-nucleon data, binding energies and scattering phase shifts.

%\end{itemize}

%\noindent
%\item Nowaydays, chiral perturbation theory gives an effective theory that allows a systematic expansion in terms of contrallable parameters. Good basis for many-body physics
\end{itemize}

\noindent
\section{The Lippman-Schwinger equation for two-nucleon scattering}

What follows now is a more technical discussion on how we can solve the two-nucleon problem.
This will lead us to the so-called Lippman-Schwinger equation for the scattering problem and a rewrite of Schroedinger's equation in relative and center-of-mass coordinates. 

\noindent
Before we break down the Schroedinger equation into a partial wave decomposition, we derive now the so-called Lippman-Schwinger equation. We will do this in an operator form first.
Thereafter, we rewrite it in terms of various quantum numbers such as relative momenta, orbital momenta etc. 
The Schroedinger equation in abstract vector representation is
\[
  \left( \hat{H}_0 + \hat{V} \right) \vert \psi_n \rangle = E_n \vert\psi_n \rangle. 
\]
In our case for the two-body problem $\hat{H}_0$ is just the kinetic energy. 
We rewrite it as 
\[
\left( \hat{H}_0 -E_n \right)\vert\psi_n \rangle =-\hat{V}\vert \psi_n \rangle . 
\]
We assume that the invers of $\left( \hat{H}_0 -E_n\right)$ exists and rewrite this equation as
\[
\vert\psi_n \rangle =\frac{1}{\left( E_n -\hat{H}_0\right)}\hat{V}\vert \psi_n \rangle . 
\]
This equation is normally solved in an iterative fashion. 
We assume first that
\[
\vert\psi_n \rangle = \vert\phi_n \rangle,
\] 
where $\vert\phi_n \rangle$ are the eigenfunctions of 
\[
\hat{H}_0\vert \phi_n \rangle=\omega_n\vert \phi_n \rangle
\]
the so-called unperturbed problem. In our case, these will simply be the kinetic energies of the relative motion. 
Inserting  $\vert\phi_n \rangle$  on the right-hand side of 
\[
\vert \psi_n \rangle =\frac{1}{( E_n -\hat{H}_0)}\hat{V}\vert \psi_n \rangle,
\]
yields
\[
\vert \psi_n \rangle =\vert\phi_n \rangle+\frac{1}{\left( E_n -\hat{H}_0\right)}\hat{V}\vert \phi_n \rangle,
\]
as our first iteration. 
Reinserting again gives
\[
\vert \psi_n \rangle =\vert\phi_n \rangle+\frac{1}{\left( E_n -\hat{H}_0\right)}\hat{V}\vert \phi_n \rangle+\frac{1}{( E_n -\hat{H}_0)}\hat{V}\frac{1}{\left( E_n -\hat{H}_0\right)}\hat{V}\vert \phi_n \rangle,
\]
and continuing we obtain
\[
\vert \psi_n \rangle =\sum_{i=0}^{\infty}\left[\frac{1}{( E_n -\hat{H}_0)}\hat{V}\right]^i\vert \phi_n \rangle.
\]
It can be rewritten as 
\[
\vert \psi_n \rangle =\vert\phi_n \rangle+\frac{1}{( E_n -\hat{H}_0)}
\hat{V}\left[1+ \frac{1}{(E_n -\hat{H}_0)}\hat{V}+\frac{1}{(E_n -\hat{H}_0)}\hat{V}\frac{1}{(E_n -\hat{H}_0)}\hat{V}+\dots\right]\vert \phi_n \rangle,
\]
which we rewrite as 
\[
\vert \psi_n \rangle =\vert\phi_n \rangle+\frac{1}{(E_n -\hat{H}_0)}\hat{V}\vert \psi_n \rangle.
\]
%In operator form we have thus
%\[
%\vert \psi_n \rangle =\vert\phi_n \rangle+\frac{1}{(E_n -\hat{H}_0)}\hat{V}\vert \psi_n \rangle.
%\]
We multiply from the left with $\hat{V}$ and $\langle \phi_m \vert$ and obtain
\[
\langle \phi_m \vert\hat{V}\vert \psi_n \rangle =\langle \phi_m \vert\hat{V}\vert\phi_n \rangle+\langle \phi_m \vert\hat{V}\frac{1}{(E_n -\hat{H}_0)}\hat{V}\vert \psi_n \rangle.
\]
We define thereafter the so-called $T$-matrix as
\[
\langle \phi_m \vert\hat{T}\vert \phi_n \rangle=\langle \phi_m \vert\hat{V}\vert \psi_n \rangle.
\]
We can rewrite our equation as
\[
\langle \phi_m \vert\hat{T}\vert \phi_n \rangle =\langle \phi_m \vert\hat{V}\vert\phi_n \rangle+\langle \phi_m \vert\hat{V}\frac{1}{(E_n -\hat{H}_0)}\hat{T}\vert \phi_n \rangle.
\]
The equation
\[
\langle \phi_m \vert\hat{T}\vert \phi_n \rangle =\langle \phi_m \vert\hat{V}\vert\phi_n \rangle+\langle \phi_m \vert\hat{V}\frac{1}{(E_n -\hat{H}_0)}\hat{T}\vert \phi_n \rangle,
\]
is called the Lippman-Schwinger equation. Inserting the completeness relation
\[ 
\mathbf{1} = \sum_n \vert \phi_n\rangle\langle \phi_n \vert, \:\: \langle \phi_n\vert \phi_{n'} \rangle = \delta_{n,n'}
\]
we have 
\[
\langle \phi_m \vert\hat{T}\vert \phi_n \rangle =\langle \phi_m \vert\hat{V}\vert\phi_n \rangle+\sum_k \langle \phi_m \vert\hat{V}\vert \phi_k\rangle\frac{1}{(E_n -\omega_k)}\langle \phi_k \vert\hat{T}\vert \phi_n \rangle,
\]
which is (when we specify the state $\vert\phi_n \rangle$) an integral equation that can actually be solved by matrix inversion easily! The unknown quantity is the $T$-matrix.

Now we wish to introduce a partial wave decomposition in order to solve the Lippman-Schwinger equation. With a partial wave decomposition we can reduce a three-dimensional integral equation to a one-dimensional one. 

\noindent
Let us continue with our Schroedinger equation in the abstract vector representation
\[
\left(T + V\right)\vert\psi_n\rangle = E_n\vert\psi_n \rangle 
\]
Here $T$ is the kinetic energy operator and $V$ is the potential operator. 
The eigenstates form a complete orthonormal set according to 
\[ 
\mathbf{1}=\sum_n\vert\psi_n\rangle\langle\psi_n\vert, \:\: \langle\psi_n\vert\psi_{n'}\rangle =\delta_{n,n'} \; .
\]
The most commonly used representations are the coordinate and
the momentum space representations. They define the completeness relations 
\begin{eqnarray*}
 \mathbf{1}&=&  \int d\mathbf{r} \:\vert\mathbf{r} \rangle \langle \mathbf{r}\vert, \:\: \langle  \mathbf{r}\vert  \mathbf{r'} \rangle = \delta ( \mathbf{r}-\mathbf{r'}) \\
\mathbf{1} &=& \int d\mathbf{k} \:\vert  \mathbf{k}\rangle \langle \mathbf{k}\vert, \:\: \langle\mathbf{k}\vert  \mathbf{k'} \rangle = \delta ( \mathbf{k}-\mathbf{k'})  \; .
\end{eqnarray*}
Here the basis states in  both $\mathbf{r}$- and $\mathbf{k}$-space are dirac-delta 
function normalized. From this it follows that the plane-wave states are given by
\[
\langle\mathbf{r}\vert\mathbf{k} \rangle =\left(\frac{1}{2\pi}\right)^{3/2}\exp\left(i\mathbf{k\cdot r} \right)
\]
which is a transformation function defining the mapping from the abstract 
$\vert\mathbf{k}\rangle$ to the abstract $\vert\mathbf{r}\rangle $ space.

\noindent
That the $\mathbf{r}$-space basis states are 
delta-function normalized follows from 
\[
\delta ( \mathbf{r}-\mathbf{r'}) = \langle \mathbf{r} \vert \mathbf{r}'\rangle = \langle \mathbf{r} \vert \mathbf{1} \vert \mathbf{r}'\rangle = \int d\mathbf{k} \langle \mathbf{r}\vert \mathbf{k} \rangle \langle \mathbf{k}\vert \mathbf{r}' \rangle =\left( {1\over 2\pi}\right)^3 \int d\mathbf{k} e^{i \mathbf{k}(\mathbf{r} - \mathbf{r}')} 
\]
and the same for the momentum space basis states,
\[
\delta ( \mathbf{k}-\mathbf{k'}) = \langle \mathbf{k} \vert \mathbf{k}'\rangle = \langle \mathbf{k} \vert \mathbf{1} \vert \mathbf{k}'\rangle =\int d\mathbf{r} \langle \mathbf{k}\vert \mathbf{r} \rangle \langle \mathbf{r}\vert \mathbf{k}' \rangle = \left( {1\over 2\pi}\right)^3 \int d\mathbf{r} e^{i \mathbf{r}(\mathbf{k} - \mathbf{k}')} 
\]
Projecting  on momentum states, we obtain the momentum space Schroedinger equation as
\begin{equation}
\frac{\hbar^2}{2\mu}k^2\psi_n(\mathbf{k})+\int d\mathbf{k'}V(\mathbf{k}, \mathbf{k'}) \psi_n(\mathbf{k'})=E_n \psi_n(\mathbf{k})
\label{eq:momspace1}
\end{equation}
Here the notation $\psi_n(\mathbf{k}) =\langle\mathbf{k}\vert\psi_n\rangle $ and 
$\langle\mathbf{k}\vert V\vert\mathbf{k}' \rangle =V(\mathbf{k}, \mathbf{k'})$ has been introduced.
The potential in momentum space is given by a double Fourier-transform 
of the potential in coordinate space, i.e.
\[ 
V(\mathbf{k},\mathbf{k'}) = \left( \frac{1}{2\pi}\right)^3\int d\mathbf{r}\int d\mathbf{r}'\exp{-i\mathbf{kr}}V(\mathbf{r},\mathbf{r}')\exp{i\mathbf{k}'\mathbf{r}'}  
\]
Here it is assumed that the potential interaction does not contain any spin dependence. 
Instead of a differential equation in coordinate space, the Schroedinger
equation becomes an integral equation in momentum space. This has 
many tractable features. Firstly, most realistic 
nucleon-nucleon interactions derived from field-theory are given 
explicitly in momentum space. Secondly, the boundary conditions imposed
on the differential equation in coordinate space are automatically built into the
integral equation. And last, but not least, integral equations are easy to numerically 
implement, and convergence is obtained by just increasing the number of integration
points.
Instead of solving the three-dimensional integral equation, an 
infinite set of 1-dimensional equations can be obtained via a  partial wave
expansion. 

\noindent
The wave function $\psi_n(\mathbf{k})$ can be expanded in a complete set of spherical harmonics, that is
\begin{equation}
  \psi_n(\mathbf{k}) = \sum _{lm} \psi_{nlm}(k)Y_{lm}(\hat{k}) \hspace{1cm} \psi_{nlm}(k) = \int d\hat{k} Y_{lm}^*(\hat{k})\psi_n(\mathbf{k}).   , 
  \label{eq:part_wave1}
\end{equation}
By inserting equation (\ref{eq:part_wave1}) in equation (\ref{eq:momspace1}), and projecting from the left
$Y_{lm}(\hat{k})$, the three-dimensional Schroedinger equation~(\ref{eq:momspace1}) is reduced
to an infinite set of  1-dimensional angular momentum coupled integral equations, 
\begin{equation}
  \left( \frac{\hbar^2}{2\mu} k^2-E_{nlm}\right)\psi_{nlm}(k) = -\sum_{l'm'}\int_{0}^\infty dk' {k'}^2 V_{lm, l'm'}(k,k') \psi_{nl'm'}(k') 
  \label{eq:part_wave2}
\end{equation}
where the angular momentum projected potential takes the form,
\begin{equation}
  V_{lm, l'm'}(k,k') = \int d{\hat{k}} \int d{\hat{k}'}Y_{lm}^*(\hat{k})V(\mathbf{k}\mathbf{k'})Y_{l'm'}(\hat{k}')
  \label{eq:pot1}
\end{equation}
here $d\hat{k} = d\theta\sin(\theta)d\varphi$.
Note that we discuss only the orbital momentum, we will include angular momentum and spin later. 



The potential is often given in position space. It is then convenient to establish 
the connection between $V_{lm, l'm'}(k,k')$ and $V_{lm, l'm'}(r,r')$. Inserting 
the completeness relation for the position quantum numbers in equation (\ref{eq:pot1}) results in
\begin{equation}
V =\int d\mathbf{r}\int d\mathbf{r}'\left\{\int d{\hat{k}}Y_{lm}^*(\hat{k})\langle \mathbf{k}\vert \mathbf{r}\rangle\right\}\langle\mathbf{r}\vert V\vert\mathbf{r}'\rangle\left\{\int d\hat{k}'Y_{lm}(\hat{k}')\langle\mathbf{r'}\vert\mathbf{k}'\rangle\right\}
\label{eq:pot2}
\end{equation}


Since the plane waves depend only on the absolute values of position and momentum, $\vert\mathbf{k}\vert$ and 
$\vert\mathbf{r}\vert$,
and the angle between them, $\theta_{kr}$, they may be expanded in terms of bipolar harmonics of 
zero rank, i.e.  
\[ 
  \exp{(i \mathbf{k}\cdot \mathbf{r})} = 4\pi\sum_{l=0}^{\infty} i^l j_l(kr)\left( Y_l(\hat{k}) \cdot Y_l(\hat{r}) \right)= \sum_{l=0}^{\infty} (2l+1)i^l j_l(kr) P_l(\cos \theta_{kr}) 
\]
where the addition theorem for spherical harmonics has been used in order to write
the expansion in terms of Legendre polynomials. The spherical Bessel functions, $j_l(z)$,  
are given in terms of Bessel functions of the first kind with half integer orders,  
\[
j_l(z) = \sqrt{\pi \over 2 z} J_{l+1/2}(z).  
\]
Inserting the plane-wave expansion
into the brackets of equation~(\ref{eq:pot2}) yields, 
\begin{eqnarray*}
  \nonumber
  \int d{\hat{k}}  Y_{lm}^*(\hat{k})\langle \mathbf{k}\vert \mathbf{r} \rangle & = &  
  \left( {1\over 2\pi} \right) ^{3/2}4\pi i^{-l} j_l(kr) Y_{lm}^*(\hat{r}), \\  
  \nonumber
  \int d{\hat{k}'}\:   Y_{lm}(\hat{k}') \langle \mathbf{r'}\vert \mathbf{k}' \rangle & = &  
  \left( {1\over 2\pi} \right) ^{3/2}4\pi i^{l'} j_{l'}(k'r') Y_{l'm'}(\hat{r}). 
\end{eqnarray*}
The connection between the momentum- and position space angular momentum 
projected potentials are then given, 
\[
  V_{lm, l'm'}(k,k')=\frac{2}{\pi}i^{l'-l}\int_0^\infty drr^2 \int_0^\infty dr'{r'}^2j_l(kr) V_{lm,l'm'}(r,r') j_{l'}(k'r')
  \label{eq:pot3}
\]
which is known as a double Fourier-Bessel transform. The position space angular 
momentum projected potential is given by
\[
  V_{lm, l'm'}(r,r') = \int d{\hat{r}} \int d{\hat{r}'}Y_{lm}^*(\hat{r})V(\mathbf{r}, \mathbf{r'})Y_{l'm'}(\hat{r}').
  \label{eq:pot4}
\]
No assumptions of locality/non-locality and deformation of the interaction has so far been made, 
and the result in equation~(\ref{eq:pot3}) is general. In position space the Schroedinger equation 
takes form of an integro-differential equation in case of a non-local interaction, 
in momentum space the Schroedinger equation is an ordinary integral equation of the Fredholm type, 
see equation~(\ref{eq:part_wave2}). This is a further advantage of the momentum space approach as compared to 
the standard position space approach.  
If we assume that the 
interaction is of local character, i.e. 
\[
  \langle \mathbf{r}\vert V \vert \mathbf{r'}\rangle = V(\mathbf{r}) \delta( \mathbf{r}-\mathbf{r}' ) = 
  V(\mathbf{r}) {\delta( { r}-{r}' ) \over r^2} \delta ( \cos \theta - \cos \theta' ) \delta (\varphi-\varphi'), 
\]
then equation~(\ref{eq:pot4}) reduces to 
\begin{equation}
  V_{lm, l'm'}(r,r') = \frac{\delta({r}-{r}')}{r^2}\int d{\hat{r}}\:
  Y_{lm}^*(\hat{r})V(\mathbf{r})Y_{l'm'}(\hat{r}),
  \label{eq:pot5}
\end{equation}
and equation (\ref{eq:pot3}) reduces to  
\begin{equation}
  V_{lm, l'm'}(k,k') = \frac{2}{\pi}i^{l' -l}\int_0^\infty drr^2j_l(kr) V_{lm,l'm'}(r) j_{l'}(k'r)
  \label{eq:pot6}
\end{equation}
where 
\begin{equation}
  V_{lm, l'm'}(r) = \int d{\hat{r}}Y_{lm}^*(\hat{r})V(\mathbf{r})Y_{l'm'}(\hat{r}),
  \label{eq:pot10}
\end{equation}
In the case that the interaction is central, $V(\mathbf{r}) = V(r)$, then
\begin{equation}
  V_{lm, l'm'}(r) = V(r) \int d{\hat{r}}Y_{lm}^*(\hat{r})Y_{l'm'}(\hat{r}) = V(r) \delta_{l,l'}\delta_{m,m'},
  \label{eq:pot7}
\end{equation}
and 
\begin{equation}
  V_{lm, l'm'}(k,k') = \frac{2}{\pi} \int_0^\infty drr^2j_l(kr) V(r) j_{l'}(k'r)\delta_{l,l'}\delta_{m,m'} = V_l(k,k') \delta_{l,l'}\delta_{m,m'}
  \label{eq:pot8}
\end{equation}
where the momentum space representation of the interaction finally reads,
\begin{equation}
  V_{l}(k,k') = {2 \over \pi} \int_0^\infty dr\: r^2 \:
  j_l(kr) V(r) j_{l}(k'r).
  \label{eq:pot9}
\end{equation}
For a local and spherical symmetric potential, 
the coupled momentum space Schroedinger equations given in equation~(\ref{eq:part_wave2})
decouples in angular momentum, 
giving
\begin{equation}
\frac{\hbar^2}{2\mu} k^2 \psi_{n l}(k) +\int_{0}^\infty dk' {k'}^2 V_{l}(k,k') \psi_{n l }(k')=E_{n l} \psi_{n l}(k) 
  \label{eq:momentum_space}
\end{equation}   
Where we have written $\psi_{n l }(k)=\psi_{nlm}(k)$, since the 
equation becomes independent of the projection $m$ for spherical symmetric interactions. 
The momentum space wave functions $\psi_{n l}(k)$ defines a complete orthogonal set 
of functions, which spans the space of functions with a positive finite Euclidean norm 
 (also called $l^2$-norm), $\sqrt{\langle\psi_n\vert\psi_n\rangle}$, which 
is a Hilbert space. The corresponding normalized wave function in coordinate space
is given by the Fourier-Bessel transform 
\[
  \phi_{n l}(r)  = \sqrt{\frac{2}{\pi}}\int dk k^2 j_l(kr) \psi_{n l}(k)
\]    
We will thus assume that the interaction is spherically symmetric and use
the partial wave expansion of the plane waves in
terms of spherical harmonics.
This means that we can separate the radial part of the wave function from its
angular dependence. The wave function of the relative motion is described
in terms of plane waves as
\[
\exp{(\imath \mathbf{kr})}=\langle\mathbf{r}\vert\mathbf{k}\rangle=4\pi\sum_{lm}\imath^{l}j_{l}(kr)Y_{lm}^{*}(\mathbf{\hat{k}})Y_{lm}(\mathbf{\hat{r}}),
\]
where $j_l$ is a spherical Bessel function and $Y_{lm}$ the
spherical harmonics.



In terms of the relative and center-of-mass momenta $\mathbf{k}$ and
$\mathbf{K}$, the potential in momentum space is related to the nonlocal operator
$V(\mathbf{r},\mathbf{r}')$ by
\[
\langle\mathbf{k'K'}\vert V \vert \mathbf{kK}\rangle =\int d\mathbf{r}d \mathbf{r'}
        \exp{-(\imath \mathbf{k'r'})}V(\mathbf{r'},\mathbf{r})\exp{\imath \mathbf{kr}}\delta(\mathbf{K},\mathbf{K'}).
\]
We will assume that the interaction is spherically symmetric.
Can separate the radial part of the wave function from its
angular dependence. The wave function of the relative motion is described
in terms of plane waves as
\[
\exp{(\imath \mathbf{kr})} =\langle\mathbf{r}\vert\mathbf{k}\rangle= 4\pi\sum_{lm}\imath^{l}j_{l}(kr)Y_{lm}^{*}(\mathbf{\hat{k}})Y_{lm}(\mathbf{\hat{r}}),
\]
where $j_l$ is a spherical Bessel function and $Y_{lm}$ the
spherical harmonic.

\noindent
This partial wave basis is useful for defining the operator for
the nucleon-nucleon interaction, which
is symmetric with respect to rotations, parity and
isospin transformations. These symmetries imply that the interaction is
diagonal with respect to the quantum numbers of total relative angular
momentum ${\cal J}$, spin $S$ and isospin $T$ (we skip isospin for the moment). Using the above plane wave expansion,
and coupling to final ${\cal J}$ and $S$ and $T$ we get
\begin{eqnarray*}
 \langle\mathbf{k'}\vert V \vert\mathbf{k}\rangle & = & (4\pi)^2 \sum_{STll'm_lm_{l'}{\cal J}}\imath^{l+l'} Y_{lm}^{*}(\mathbf{\hat{k}}) Y_{l'm'}(\mathbf{\hat{k}'})\\ & ~ &
\times \langle lm_lSm_S|{\cal J}M\rangle \langle l'm_{l'}Sm_S|{\cal J}M\rangle\langle k'l'S{\cal J}M\vert V \vert klS{\cal J}M\rangle,
\end{eqnarray*}
where we have defined
\[
    \langle k'l'S{\cal J}M\vert V \vert klS{\cal J}M\rangle=\int j_{l'}(k'r')\langle l'S{\cal J}M\vert V(r',r)\vert lS{\cal J}M\rangle j_l(kr) {r'}^2 dr' r^2 dr.
\]
We have omitted the momentum of the center-of-mass motion $\mathbf{K}$ and the 
corresponding orbital momentum $L$, since the interaction is diagonal
in these variables.

\noindent
We wrote the Lippman-Schwinger equation as
\[
\langle \phi_m \vert\hat{T}\vert \phi_n \rangle =\langle \phi_m \vert\hat{V}\vert\phi_n \rangle+\sum_k \langle \phi_m \vert\hat{V}\vert \phi_k\rangle\frac{1}{(E_n -\omega_k)}\langle \phi_k \vert\hat{T}\vert \phi_n \rangle.
\]
How do we rewrite it in a partial wave expansion with momenta $k$? The general structure of the $T$-matrix in partial waves is
\begin{eqnarray*}
   T_{ll'}^{\alpha}(kk'K\omega) & =& V_{ll'}^{\alpha}(kk')
   +{\displaystyle \frac{2}{\pi}\sum_{l''m_{l''}M_S}\int_{0}^{\infty} d \mathbf{q}
   (\langle l''m_{l''}Sm_S|{\cal J}M\rangle)^2 }\\
   & ~ & {\displaystyle \times
   \frac{Y_{l''m_{l''}}^*(\hat{\mathbf{q}})
   Y_{l''m_{l''}}(\hat{\mathbf{q}}) V_{ll''}^{\alpha}(kq)
   T_{l''l'}^{\alpha}(qk'K\omega)}
   {\omega -H_0}} \; .
   \label{eq:bspartial}
\end{eqnarray*}
The  shorthand notation
\[
    T_{ll'}^{\alpha}(kk'K\omega)=
   \langle kKlL{\cal J}S\vert T(\omega)\vert k'Kl'L{\cal J}S\rangle,
\]
denotes the $T$-matrix
with momenta $k$ and $k'$ and orbital momenta $l$ and $l'$
of the relative motion, and
$K$ is the corresponding momentum of
the center-of-mass motion. Further, $L$, ${\cal J}$, $S$ and $T$
are the orbital momentum of the center-of-mass motion, the
total angular momentum,
spin and isospin, respectively. 
Due to the nuclear tensor force, the interaction is not diagonal in $ll'$.


Using the orthogonality
properties of the Clebsch-Gordan coefficients and the spherical harmonics,
we obtain the well-known
one-dimensional angle independent
integral equation
\[
   T_{ll'}^{\alpha}(kk'K\omega)=V_{ll'}^{\alpha}(kk')
   +\frac{2}{\pi}\sum_{l''}\int_{0}^{\infty} dqq^2
   \frac{V_{ll''}^{\alpha}(kq)
   T_{l''l'}^{\alpha}(qk'K\omega)}
   {\omega -H_0}.
\]
Inserting the denominator we arrive at 
\[
   \hat{T}_{ll'}^{\alpha}(kk'K)=\hat{V}_{ll'}^{\alpha}(kk')
   +\frac{2}{\pi}\sum_{l''}\int_{0}^{\infty} dqq^2
   \hat{V}_{ll''}^{\alpha}(kq)
   \frac{1}{k^2-q^2 +i\epsilon}
   \hat{T}_{l''l'}^{\alpha}(qk'K).
\]
To parameterize the nucleon-nucleon interaction we solve the Lippman-Scwhinger
equation
\[
   T_{ll'}^{\alpha}(kk'K)=V_{ll'}^{\alpha}(kk')
   +\frac{2}{\pi}\sum_{l''}\int_{0}^{\infty} dqq^2
   V_{ll''}^{\alpha}(kq)
   \frac{1}{k^2-q^2 +i\epsilon}
   T_{l''l'}^{\alpha}(qk'K).
\]
The  shorthand notation
\[
    T(\hat{V})_{ll'}^{\alpha}(kk'K\omega)=\langle kKlL{\cal J}S\vert T(\omega)\vert k'Kl'L{\cal J}S\rangle,
\]
denotes the $T(V)$-matrix
with momenta $k$ and $k'$ and orbital momenta $l$ and $l'$
of the relative motion, and
$K$ is the corresponding momentum of
the center-of-mass motion. Further, $L$, ${\cal J}$, and $S$
are the orbital momentum of the center-of-mass motion, the
total angular momentum and
spin, respectively. We skip for the moment isospin.


For scattering states, the energy is positive, $E>0$. 
The Lippman-Schwinger equation (a rewrite of the Schroedinger equation)
is an integral equation
where we have to deal with the amplitude 
$R(k,k')$ (reaction matrix, which is the real part of  the full
complex $T$-matrix)
defined through the integral equation for one partial wave (no coupled-channels) 
\begin{equation}
    R_l(k,k') = V_l(k,k') +\frac{2}{\pi}{\cal P}
                \int_0^{\infty}dqq^2V_l(k,q)\frac{1}{E-q^2/m}R_l(q,k').
   \label{eq:ls1}
\end{equation}
For negative energies (bound states) and intermediate states scattering states blocked
by  occupied states below the Fermi level.


The symbol ${\cal P}$ in the previous slide indicates that Cauchy's principal-value prescription
is used in order to avoid the singularity arising from the zero of the denominator.


The total kinetic energy of the two 
incoming particles in the center-of-mass system
is 
\[
    E=\frac{k_0^2}{m_n}.
\]



The matrix $R_l(k,k')$ relates to the 
the  phase shifts through its diagonal elements as
\begin{equation}
     R_l(k_0,k_0)=-\frac{tan\delta_l}{mk_0}.
     \label{eq:shifts}
\end{equation}
From now on we will drop the subscript $l$ in all equations.

\noindent
In order to solve the Lippman-Schwinger equation 
in momentum space, we need first to write 
a function which sets up the mesh points. 
We need to do that since we are going to approximate an integral
through 
\[
   \int_a^bf(x)dx\approx\sum_{i=1}^Nw_if(x_i),
\]
where we have fixed $N$ lattice points through the corresponding weights
$w_i$ and points $x_i$. Typically obtained via methods like Gaussian quadrature.

If you use Gauss-Legendre the points are determined for the interval $x_i\in [-1,1]$
You map these points over to the limits in your integral. You can then
use the following mapping
\[
  k_i=const\times tan\left\{\frac{\pi}{4}(1+x_i)\right\},
\]
and 
\[
   \omega_i= const\frac{\pi}{4}\frac{w_i}{cos^2\left(\frac{\pi}{4}(1+x_i)\right)}.
\]
If you choose units fm$^{-1}$ for $k$, set $const=1$. If you choose to work
with MeV, set $const\sim 200$ ($\hbar c=197$ MeVfm).


The principal value integral is rather tricky
to evaluate numerically, mainly since computers have limited
precision. We will here use a subtraction trick often used
when dealing with singular integrals in numerical calculations.
We introduce first the calculus relation
\[
  \int_{-\infty}^{\infty} \frac{dk}{k-k_0} =0.
\]
It means that the curve $1/(k-k_0)$ has equal and opposite
areas on both sides of the singular point $k_0$. If we break
the integral into one over positive $k$ and one over 
negative $k$, a change of variable $k\rightarrow -k$ 
allows us to rewrite the last equation as
\[
  \int_{0}^{\infty} \frac{dk}{k^2-k_0^2} =0.
\]
We can then express a principal values integral
as
\begin{equation}
  {\cal P}\int_{0}^{\infty} \frac{f(k)dk}{k^2-k_0^2} =
  \int_{0}^{\infty} \frac{(f(k)-f(k_0))dk}{k^2-k_0^2},
   \label{eq:trick}
\end{equation}
where the right-hand side is no longer singular at 
$k=k_0$, it is proportional to the derivative $df/dk$,
and can be evaluated numerically as any other integral.

\noindent
We can then use this trick to obtain
\begin{equation}
    R(k,k') = V(k,k') +\frac{2}{\pi}
                \int_0^{\infty}dq
                \frac{q^2V(k,q)R(q,k')-k_0^2V(k,k_0)R(k_0,k')  }
                     {(k_0^2-q^2)/m}.
   \label{eq:ls2}
\end{equation}
This is the equation to solve numerically in order
to calculate the phase shifts. We are interested in obtaining
$R(k_0,k_0)$.  Using the mesh points $k_j$ and the weights $\omega_j$, we reach
\begin{eqnarray*}
          R(k,k') & = & V(k,k') +\frac{2}{\pi}
          \sum_{j=1}^N\frac{\omega_jk_j^2V(k,k_j)R(k_j,k')}
                           {(k_0^2-k_j^2)/m} \\
& ~ &            -\frac{2}{\pi}k_0^2V(k,k_0)R(k_0,k')
          \sum_{n=1}^N\frac{\omega_n}
                           {(k_0^2-k_n^2)/m}.                
\end{eqnarray*}
This equation contains now the unknowns $R(k_i,k_j)$
(with dimension $N\times N$) and $R(k_0,k_0)$.

We can turn it into an equation
with dimension $(N+1)\times (N+1)$ with  a mesh
which contains the original mesh points $k_j$ for $j=1,N$
and the point which corresponds to the energy $k_0$.
Consider the latter as the 'observable' point.
The mesh points become then $k_j$ for $j=1,n$ and
$k_{N+1}=k_0$. 

With these new mesh points we define the matrix
\begin{equation}
      A_{i,j}=\delta_{i,j}-V(k_i,k_j)u_j,
      \label{eq:aeq}
\end{equation}
where $\delta$ is the Kronecker $\delta$
and
\[
     u_j=\frac{2}{\pi}\frac{\omega_jk_j^2}{(k_0^2-k_j^2)/m}\hspace{1cm} j=1,N
\]
and
\[
     u_{N+1}=-\frac{2}{\pi}\sum_{j=1}^N\frac{k_0^2\omega_j}{(k_0^2-k_j^2)/m}.
\]
The first task is then to 
set up the matrix $A$ for a given $k_0$. This is an
$(N+1)\times (N+1)$ matrix. It can be convenient
to have an outer loop which runs over the chosen
observable values for the energy $k_0^2/m$.
{\em Note that all mesh points $k_j$ for $j=1,N$ must be
different from $k_0$. Note also that
$V(k_i,k_j)$ is an
$(N+1)\times (N+1)$ matrix}. 

With the matrix $A$ we can rewrite the problem as a matrix problem of dimension $(N+1)\times (N+1)$.
All matrices $R$, $A$ and $V$ have this dimension and we get
\[
    A_{i,l}R_{l,j}=V_{i,j},
\] 
or just
\[
    AR=V.
\] 
Since you already have defined $A$ and $V$
(these are stored as $(N+1)\times (N+1)$ matrices) 
The final equation involves only the unknown
$R$. We obtain it by matrix inversion, i.e.,
\begin{equation}
    R=A^{-1}V.
    \label{eq:final2}
\end{equation}
Thus, to obtain $R$, you will need to set up the matrices
$A$ and $V$ and invert the matrix $A$. 
With the inverse $A^{-1}$, perform
a matrix multiplication with $V$ results in $R$.

With $R$ you can then evaluate the phase shifts
by noting that 
\[
      R(k_{N+1},k_{N+1})=R(k_0,k_0)=-\frac{tan\delta}{mk_0},
\]
where $\delta$ are the phase shifts.

For elastic scattering, the scattering potential can only change the outgoing spherical wave function up to a phase. In the asymptotic limit, far away from the scattering potential, we get for the spherical bessel function
\[
j_l(kr) \xrightarrow[]{ r \gg 1} \frac{\sin(kr -l\pi/2)}{kr} =  \frac{1}{2ik}\left( \frac{e^{i(kr-l\pi/2)}}{r} - \frac{e^{-i(kr-l\pi/2)}}{r}\right)
\]
The outgoing wave will change by a phase shift $\delta_l$, from which we can define the S-matrix $S_l(k) = e^{2i\delta_l(k)}$. Thus, we have
\[
 \frac{e^{i(kr-l\pi/2)}}{r} \xrightarrow[]{\mathrm{phase~change}}  \frac{S_l(k)e^{i(kr-l\pi/2)}}{r}
\]
The solution to the Schrodinger equation for a spherically symmetric potential, will have the form
\[
\psi_k(r) = e^{ikr} + f(\theta)\frac{e^{ikr}}{r}
\]
where $f(\theta)$ is the scattering amplitude, and related to the differential cross section as
\[
\frac{d\sigma}{d\Omega} = |f(\theta)|^2
\]
Using the expansion of a plane wave in spherical waves, we can relate the scattering amplitude $f(\theta)$ with the partial wave phase shifts $\delta_l$ by identifying the outgoing wave 
\[
\psi_k(r) = e^{ikr} + \left[\frac{1}{2ik}\sum_l i^l (2l+1) (S_l(k)-1)P_l(\cos(\theta))e^{-il\pi/2}\right] \frac{e^{ikr}}{r}
\]
which can be simplified further by cancelling $i^l$ with $e^{-il\pi/2}$. We have
\[
\psi_k(r) = e^{ikr} + f(\theta) \frac{e^{ikr}}{r}
\]
with 
\[
f(\theta) = \sum_l (2l+1)f_l(\theta) P_l(\cos(\theta))
\]
where the partial wave scattering amplitude is given by
\[
f_l(\theta) = \frac{1}{k}\frac{(S_l(k)-1)}{2i} = \frac{1}{k}\sin\delta_l(k) e^{i\delta_l(k)}
\]
With Eulers formula for the cotangent, this can also be written as
\[
f_l(\theta) = \frac{1}{k}\frac{1}{\cot \delta_l(k) - i}.
\]


\begin{figure}[t]
  \centerline{\includegraphics[width=1.2\linewidth]{phase.png}}
  
  \vspace{.5cm}
  \caption{
  Examples of negative and positive phase shifts for repulsive and attractive potentials, respectively.
  }
\end{figure}
%\clearpage % flush figures 


\noindent
The integrated cross section is given by
\begin{eqnarray*}
\sigma & = & 2\pi \int_0^{\pi} |f(\theta)|^2 \sin \theta d\theta \\ & = & 
2\pi \sum_l |\frac{(2l+1)}{k} \sin(\delta_l)|^2 \int_0^{\pi} (P_l(\cos(\theta)))^2 \sin(\theta) d\theta \\
& = & \frac{4\pi}{k^2} \sum_l (2l+1) \sin^2\delta_l(k) = 4\pi \sum_l (2l+1)|f_l(\theta)|^2, 
\end{eqnarray*}
where the orthogonality of the Legendre polynomials was used to evaluate the last integral
\[
\int_0^{\pi} P_l(\cos \theta)^2 \sin \theta d\theta = \frac{2}{2l+1}.
\]
Thus, the \textbf{total} cross section is the sum of the partial-wave cross sections. Note that the differential cross section contains cross-terms from different partial waves. The integral over the full sphere enables the use of the legendre orthogonality, and this kills the cross-terms.

At low energy, $k \rightarrow 0$, S-waves are most important. In this region we can define the scattering length $a$ and the effective range $r$. The $S-$wave scattering amplitude is given by
\[
f_l(\theta) = \frac{1}{k}\frac{1}{\cot \delta_l(k) - i}.
\]
Taking the limit $k \rightarrow 0$, gives us the expansion
\[
k \cot \delta_0 = -\frac{1}{a} + \frac{1}{2}r_0 k^2 + \ldots
\]
Thus the low energy cross section is given by
\[
\sigma = 4\pi a^2.
\]
If the system contains a bound state, the scattering length will become positive (neutron-proton in $^3S_1$). For the $^1S_0$ wave, the scattering length is negative and large. This indicates that the wave function of the system is at the verge of turning over to get a node, but cannot create a bound state in this wave.

\begin{figure}[t]
  \centerline{\includegraphics[width=1.0\linewidth]{sclength.png}}
  
    \vspace{.5cm}
  \caption{
  Examples of scattering lengths.
  }
\end{figure}
%\clearpage % flush figures 

It is important to realize that the phase shifts themselves are not
observables. The measurable scattering quantity is the cross section,
or the differential cross section. The partial wave phase shifts can
be thought of as a parameterization of the (experimental) cross
sections. The phase shifts provide insights into the physics of
partial wave projected nuclear interactions, and are thus important
quantities to know.

The nucleon-nucleon differential cross section
have been measured at almost all energies up to the pion production
threshold (290 MeV in the Lab frame), and this experimental data base
is what provides us with the constraints on our nuclear interaction
models. In order to pin down the unknown coupling constants of the
theory, a statistical optimization with respect to cross sections need
to be carried out. 

%\begin{figure}[t]
 % \centerline{\includegraphics[width=1.0\linewidth]{fig-ch5/nijmegen_pp_phase_shifts.png}}
  %\caption{
  %Nijmegen phase shifts for selected partial waves.
 % }
%\end{figure}
%\clearpage % flush figures 

The $pp$-data is more accurate than the $np$-data, and for $nn$ there is no data. The quality of a potential is gauged by the $\chi^2$/datum with respect to the scattering data base

{  % Springer T2 style: small table font and more vspace

\vspace{4mm}

\begin{tabular}{ccccc}
\hline
\multicolumn{1}{c}{ $T_{\mathrm{lab}}$ bin (MeV) } & \multicolumn{1}{c}{ N3LO$^1$ } & \multicolumn{1}{c}{ NNLO$^2$ } & \multicolumn{1}{c}{ NLO$^2$ } & \multicolumn{1}{c}{ AV18$^3$ } \\
\hline
                             &                 &               &               &                 \\
0-100                        & 1.05            & 1.7           & 4.5           & 0.95            \\
100-190                      & 1.08            & 22            & 100           & 1.10            \\
190-290                      & 1.15            & 47            & 180           & 1.11            \\
$\mathbf{0-290}$             & $\mathbf{1.10}$ & $\mathbf{20}$ & $\mathbf{86}$ & $\mathbf{1.04}$ \\
\hline
\end{tabular}

\vspace{4mm}

}


\clearpage % flush figures 

% --- begin exercise ---
\begin{doconceexercise}
\refstepcounter{doconceexercisecounter}

\subsection*{Exercise \thedoconceexercisecounter: Find all possible two-body quantum numbers}
\addcontentsline{loe}{doconceexercise}{Exercise \thedoconceexercisecounter: Find all possible two-body quantum numbers}


List all \textbf{allowed} according to the Pauli principle partial waves with isospin $T$, their 
projection $T_z$, spin $S$, orbital angular momentum $l$ and total spin $J$ for $J\le 3$.
Use the standard spectroscopic notation $^{2S+1}L_J$ to label different partial waves. A proton-proton state
has $T_Z=-1$, a proton-neutron state has $T_z=0$ and a neutron-neutron state has $T_z=1$.

\end{doconceexercise}
% --- end exercise ---




% --- begin exercise ---
\begin{doconceexercise}
\refstepcounter{doconceexercisecounter}

\subsection*{Exercise \thedoconceexercisecounter: Expressions for the various operators}
\addcontentsline{loe}{doconceexercise}{Exercise \thedoconceexercisecounter: Expressions for the various operators}



\subex{a)}
Find the closed form expression for the spin-orbit force. Show that the spin-orbit force {\bf LS} gives a zero
contribution for $S$-waves (orbital angular momentum $l=0$).   What is the value of the spin-orbit force for spin-singlet states ($S=0$)?

\subex{b)}
Find thereafter the expectation value of $\mathbf{\sigma}_1\cdot\mathbf{\sigma}_2$, where $\mathbf{\sigma}_i$ are so-called Pauli matrices. 
\begin{enumerate}
\item Add thereafter isospin and find the expectation value of $\mathbf{\sigma}_1\cdot\mathbf{\sigma}_2\mathbf{\tau}_1\cdot\mathbf{\tau}_2$, where $\mathbf{\tau}_i$ are also so-called Pauli matrices. List all the cases with $S=0,1$ and $T=0,1$.
\end{enumerate}

\noindent
\end{doconceexercise}
% --- end exercise ---




% --- begin exercise ---
\begin{doconceexercise}
\refstepcounter{doconceexercisecounter}

\subsection*{Exercise \thedoconceexercisecounter: One-pion exchange}
\addcontentsline{loe}{doconceexercise}{Exercise \thedoconceexercisecounter: One-pion exchange}


A simple parametrization of the nucleon-nucleon force is given by what is called the $V_8$ potential model,
where we have kept eight different operators. These operators contain a central force, a spin-orbit force,
a spin-spin force and a tensor force. Several features of the nuclei can be explained in terms of these four components. Without the Pauli matrices for isospin the final form of such an interaction model results in the following form: 
\[
V(\mathbf{r})= \left\{ C_c + C_\mathbf{\sigma} \mathbf{\sigma}_1\cdot\mathbf{\sigma}_2
 + C_T \left( 1 + {3\over m_\alpha r} + {3\over\left(m_\alpha r\right)^2}\right) S_{12} (\hat r)\right. 
\]
\[
\left. + C_{SL} \left( {1\over m_\alpha r} + {1\over \left( m_\alpha r\right)^2}
\right) \mathbf{L}\cdot \mathbf{S}
\right\} \frac{e^{-m_\alpha r}}{m_\alpha r}
\]
where $m_{\alpha}$ is the mass of the relevant meson and
$S_{12}$ is the familiar tensor term. The various coefficients $C_i$ are normally fitted so that the potential reproduces experimental scattering cross sections. By adding terms which include the isospin Pauli matrices 
results in an interaction model with eight operators.

The expectaction value of the tensor operator is non-zero only for $S=1$. We will show this in a forthcoming lecture, after that we have derived the Wigner-Eckart theorem. 
Here it suffices to know that the expectaction value of the tensor force for different partial values is  (with $l$ the orbital angular momentum and ${\cal J}$ the total angular momentum in the relative and center-of-mass frame of motion)
\[
\langle l {\cal J}S=1| S_{12} | l' {\cal J}S=1\rangle = -\frac{2{\cal J}({\cal J}+2)}{2{\cal J}+1} \hspace{0.5cm} l= {\cal J}+1 \hspace{0.1cm}\mathrm{and} \hspace{0.1cm} l'={\cal J}+1,
\]
\[
\langle l {\cal J}S=1| S_{12} | l' {\cal J}S=1\rangle = \frac{6\sqrt{{\cal J}({\cal J}+1)}}{2{\cal J}+1} \hspace{0.5cm} l= {\cal J}+1 \hspace{0.1cm}\mathrm{and} \hspace{0.1cm} l'={\cal J}-1,
\]
\[
\langle l {\cal J}S=1| S_{12} | l' {\cal J}S=1\rangle = \frac{6\sqrt{{\cal J}({\cal J}+1)}}{2{\cal J}+1} \hspace{0.5cm} l= {\cal J}-1 \hspace{0.1cm}\mathrm{and} \hspace{0.1cm} l'={\cal J}+1,
\]
\[
\langle l {\cal J}S=1| S_{12} | l' {\cal J}S=1\rangle = -\frac{2({\cal J}-1)}{2{\cal J}+1} \hspace{0.5cm} l= {\cal J}-1 \hspace{0.1cm}\mathrm{and} \hspace{0.1cm} l'={\cal J}-1,
\]
\[
\langle l {\cal J}S=1| S_{12} | l' {\cal J}S=1\rangle = 2 \hspace{0.5cm} l= {\cal J} \hspace{0.1cm}\mathrm{and} \hspace{0.1cm} l'={\cal J},
\]
and zero else.   

In this exercise we will focus only on the one-pion exchange term of the nuclear force, namely
\[
V_{\pi}(\mathbf{r})= -\frac{f_{\pi}^{2}}{4\pi m_{\pi}^{2}}\mathbf{ \tau}_1\cdot\mathbf{\tau}_2
\frac{1}{3}\left\{\mathbf{ \sigma}_1\cdot\mathbf{ \sigma}_2+\left( 1 + {3\over m_\pi r} + {3\over\left(m_\pi r\right)^2}\right) S_{12} (\hat r)\right\} \frac{e^{-m_\pi r}}{m_\pi r}.
\]
Here the constant $f_{\pi}^{2}/4\pi\approx 0.08$ and the mass of the pion is $m_\pi\approx 140$ MeV/c${}^{2}$.


\subex{a)}
Compute the expectation value of the tensor force and the spin-spin  and isospin operators for the one-pion exchange potential for all partial waves you found in exercise 9. Comment your results. How does the one-pion exchange part behave as function of different $l$, ${\cal J}$ and $S$ values? Do you see some patterns?

\subex{b)}
For the binding energy of the deuteron, with the ground state defined by the quantum numbers $l=0$, $S=1$ and ${\cal J}=1$, the tensor force plays an important role due to the admixture from the $l=2$ state. Use the expectation values of the different operators of the one-pion exchange potential and plot the ratio of the tensor force component over the spin-spin component of the one-pion exchange part as function of $x=m_\pi r$ for the $l=2$ state (that is the case $l,l'={\cal J}+1$). Comment your results.





\end{doconceexercise}
% --- end exercise ---


\begin{doconceexercise}
\refstepcounter{doconceexercisecounter}

\subsection*{Exercise \thedoconceexercisecounter: Program for the Lippman-Schwinger equation}
\addcontentsline{loe}{doconceexercise}{Exercise \thedoconceexercisecounter: Program for the Lippman-Schwinger equation}


The aim here is to develop a program which solves the Lippman-Schwinger equation for a simple parametrization for the 
$^1S_0$ partial wave. This partial wave is given by a central force only and is parametrized in coordinate space as
!bt
\[
  V(r)=V_a \frac{e^{-ax}}{x}+V_b \frac{e^{-bx}}{x}+V_c \frac{e^{-cx}}{x}
\]
with $x=\mu r$, $\mu=0.7$ fm (the inverse of the pion mass),
$V_a=-10.463$ MeV and $a=1$, $V_b=-1650.6$ MeV and $b=4$ and
$V_c=6484.3$ MeV and $c=7$. 

Find an analytical expression for 
the Fourier-Bessel transform (Hankel transform)
to momentum space for $l=0$ using
!bt
\[
\left\langle k \right | V_{l} \left | k' \right\rangle
= \int j_l(kr)V(r)j_l(k'r)r^2dr.
\]
Write a small program which calculates the latter expression
and use this potential to compute the $T$-matrix at positive energies
for $l=0$.
Compare your results to those obtained with a  box potential given by
\[
V(r)=\left\{ \begin{array}{cc} -V_0& r < R_0 \\
                                0  & r > R_0 \end{array} \right.
\]
Make a plot of the
two $T$-matrices for energies up to 300 MeV in the lab frame
and comment your results.

Finally, a warning, the above central potential is fitted  
to data from approximately 
20 MeV to some 300 MeV. This means that results outside
the data set should not be taken seriously.

The following Fortran program {\tt tmatrix.f} solves the above Lippmann-Schwinger equation. 
\begin{comment}
\begin{lstlisting}[language=fortran]
C     *******************************************************
C         Example program used to evaluate the 
C         T-matrix following Kowalski's method (eqs V88 & V89
C         in Brown and Jackson)  
C         for positive energies only
C         The program is set up for S-waves only
C         Coded by : Morten Hjorth-Jensen
C         Language : FORTRAN 90
C     *******************************************************



C               ******************************
C                 Def of global variables
C               ******************************


      MODULE constants
         DOUBLE PRECISION , PUBLIC :: p_mass, hbarc
         PARAMETER (p_mass =938.926D0, hbarc = 197.327D0)
      END MODULE constants

      MODULE mesh_variables           
         INTEGER, PUBLIC :: n_rel
         PARAMETER(n_rel=48)
         DOUBLE PRECISION, ALLOCATABLE, PUBLIC :: ra(:), wra(:)
      END MODULE  mesh_variables


C               ******************************
C                   Begin of main program
C               ******************************


      PROGRAM t_matrix
      USE mesh_variables
      IMPLICIT NONE
      INTEGER istat

      ALLOCATE( ra (n_rel), wra (n_rel),  STAT=istat )
      CALL rel_mesh               ! rel mesh & weights
      CALL t_channel              ! calculate the T-matrix
      DEALLOCATE( ra,wra, STAT=istat )

      END PROGRAM t_matrix

C     *********************************************************
C                    obtain the t-mtx
C                    vkk is the box potential
C                    f_mtx is equation V88 og Brown & Jackson
C     *********************************************************

      SUBROUTINE t_channel
      USE mesh_variables
      IMPLICIT NONE
      INTEGER istat, i,j
      DIMENSION vkk(:,:),f_mtx(:),t_mtx(:)
      DOUBLE PRECISION, ALLOCATABLE :: vkk,t_mtx,f_mtx
      DOUBLE PRECISION t_shell

      ALLOCATE(vkk (n_rel,n_rel), STAT=istat)
      CALL v_pot_yukawa(vkk) ! set up the box potential in routine vpot
      ALLOCATE(t_mtx (n_rel), STAT=istat) ! allocate space in heap for T
      ALLOCATE(f_mtx (n_rel), STAT=istat) ! allocate space for f
      DO i=1,n_rel               ! loop over positive energies e=k^2
         CALL f_mtx_eq(f_mtx,vkk,i)  ! solve eq. V88
         CALL principal_value(vkk,f_mtx,i,t_shell) ! solve Eq. V89 
         DO j=1,n_rel    ! the t-matrix
            t_mtx(j)=f_mtx(j)*t_shell
            IF(j == i) WRITE(6,*) ra(i) ,t_mtx(i)
c     &                            DATAN(-ra(i)*t_mtx(i))
         ENDDO
      ENDDO
      DEALLOCATE(vkk ,  STAT=istat)
      DEALLOCATE(t_mtx, f_mtx,  STAT=istat)
 1000 FORMAT( I3, 2F12.6) 

      END SUBROUTINE t_channel

C     ***********************************************************
C          The analytical expression for the box potential
C          of exercise 1 and 12
C          vkk is in units of fm^-2 (14 MeV/41.47Mevfm^2, where 
C          41.47= \hbarc^2/mass_nucleon),  
C          ra are mesh points in rel coordinates, units of fm^-1
C     ***********************************************************

      SUBROUTINE v_pot_box(vkk)
      USE mesh_variables
      USE constants
      IMPLICIT NONE
      INTEGER i,j
      DOUBLE PRECISION  vkk, r_0, v_0, a, b, fac
      PARAMETER(r_0=2.7d0,v_0=0.33759d0)  !r_0 in fm, v_0 in fm^-2 
      DIMENSION vkk(n_rel,n_rel)

      DO i=1,n_rel     ! set up the free potential
         DO j=1,i-1 
            a=ra(i)+ra(j)
            b=ra(i)-ra(j)
            fac=v_0/(2.d0*ra(i)*ra(j))
            vkk(j,i)=fac*(DSIN(a*r_0)/a-DSIN(b*r_0)/b)
            vkk(i,j)=vkk(j,i)
         ENDDO
         fac=v_0/(2.d0*(ra(i)**2))
         vkk(i,i)=fac*(DSIN(2.d0*ra(i)*r_0)/(2.d0*ra(i))-r_0)
      ENDDO

      END  SUBROUTINE v_pot_box

C     ***********************************************************
C          The analytical expression for  a Yukawa potential
C          in the l=0 channel
C          vkk is in units of fm^-2,  
C          ra are mesh points in rel coordinates, units of fm^-1
C          The parameters here are those of the Reid-Soft core
C          potential, see Brown and Jackson eq. A(4)
C     ***********************************************************

      SUBROUTINE v_pot_yukawa(vkk)
      USE mesh_variables
      USE constants
      IMPLICIT NONE
      INTEGER i,j
      DOUBLE PRECISION  vkk, mu1, mu2, mu3, v_1, v_2, v_3, a, b, fac
      PARAMETER(mu1=0.49d0,v_1=-0.252d0) 
      PARAMETER(mu2=7.84d0,v_2=-39.802d0) 
      PARAMETER(mu3=24.01d0,v_3=156.359d0) 
      DIMENSION vkk(n_rel,n_rel)

      DO i=1,n_rel     ! set up the free potential
         DO j=1,i 
            a=(ra(j)+ra(i))**2
            b=(ra(j)-ra(i))**2
            fac=1./(4.d0*ra(i)*ra(j))
            vkk(j,i)=v_1*fac*DLOG((a+mu1)/(b+mu1))+
     &               v_2*fac*DLOG((a+mu2)/(b+mu2))+
     &               v_3*fac*DLOG((a+mu3)/(b+mu3))
            vkk(i,j)=vkk(j,i)
         ENDDO
      ENDDO

      END  SUBROUTINE v_pot_yukawa
      
C     **************************************************
C         Solves eq. V88 
C         and returns < p | f_mtx | n_pole =k>
C     **************************************************

      SUBROUTINE f_mtx_eq(f_mtx,vkk,n_pole)
      USE mesh_variables
      USE constants
      IMPLICIT NONE
      INTEGER i, j, int, istat, n_pole
      DOUBLE PRECISION f_mtx,vkk,dp,deriv,pih,xsum
      DIMENSION dp(1),deriv(1)
      DIMENSION f_mtx(n_rel),vkk(n_rel,n_rel),a(:,:),fu(:),q(:),au(:)
      DOUBLE PRECISION, ALLOCATABLE :: fu, q, au, a

      pih=2.D0/ACOS(-1.D0)
      ALLOCATE( a (n_rel,n_rel), STAT=istat)
      DO i=1,n_rel
         ALLOCATE(fu(n_rel), q(n_rel), au(n_rel), STAT=istat)
         DO j=1,n_rel
            fu(j)=vkk(i,j)-vkk(i,n_pole)*vkk(n_pole,j)/
     &                 vkk(n_pole,n_pole)
         ENDDO
         DO j=1,n_rel
            IF(j /= n_pole ) THEN     ! regular part
               a(j,i)=pih*fu(j)*wra(j)*(ra(j)**2)/
     &                (ra(j)**2-ra(n_pole)**2)
            ELSEIF(j == n_pole) THEN  ! use l'Hopitals rule to get pole term
               dp(1)=ra(j)             
               CALL spls3(ra,fu,n_rel,dp,deriv(1),1,q,au,2,0) 
               a(j,i)=pih*wra(j)*ra(j)/2.d0*deriv(1)
            ENDIF
         ENDDO
         DEALLOCATE(fu, q, au, STAT=istat)   ! free space in heap 
         a(i,i)=a(i,i)+1.D0
      ENDDO
      CALL matinv(a, n_rel)      ! Invert the matrix a
      DO j=1,n_rel               ! multiply inverted matrix a with dim less pot
         xsum=0.D0
         DO i=1,n_rel
            xsum=xsum+a(i,j)*vkk(i,n_pole)/vkk(n_pole,n_pole)  ! gives f-matrix in V88
         ENDDO
         f_mtx(j)=xsum
      ENDDO
      DEALLOCATE (a, STAT=istat)

      END SUBROUTINE f_mtx_eq

C     **************************************************
C         Solves the principal value integral of V89
C         returns the t-matrix for k=k, t_shell
C     **************************************************

      SUBROUTINE principal_value(vkk,f_mtx,n_pole,t_shell) 
      USE mesh_variables
      IMPLICIT NONE
      DOUBLE PRECISION vkk, f_mtx, t_shell, sum, pih, deriv, term
      DIMENSION deriv(1)
      DIMENSION vkk(n_rel, n_rel), f_mtx(n_rel),fu(:), q(:), au(:)
      DOUBLE PRECISION, ALLOCATABLE :: fu, q, au
      INTEGER n_pole, i, istat

      ALLOCATE(fu(n_rel), q(n_rel), au(n_rel), STAT=istat)
      sum=0.D0
      pih=2.D0/ACOS(-1.D0)
      DO i=1,n_rel
         fu(i)=vkk(n_pole,i)*f_mtx(i)
      ENDDO
      DO i=1,n_pole-1  ! integrate up to the pole - 1 mesh
         term=fu(i)*(ra(i)**2)-fu(n_pole)*(ra(n_pole)**2)
         sum=sum+pih*wra(i)*term/(ra(i)**2-ra(n_pole)**2)
      ENDDO       ! here comes the pole part
      CALL spls3(ra,fu,n_rel,ra(n_pole),deriv,1,au,q,2,0)      
      sum=sum+pih*wra(n_pole)*(fu(n_pole)+ra(n_pole)*deriv(1)/2.d0) 
      DO i=n_pole+1,n_rel  ! integrate from pole + 1mesh pt to infty
         term=fu(i)*(ra(i)**2)-fu(n_pole)*(ra(n_pole)**2)
         sum=sum+pih*wra(i)*term/(ra(i)**2-ra(n_pole)**2)
      ENDDO
      t_shell=vkk(n_pole,n_pole)/(1.d0+sum)
      DEALLOCATE (fu, q, au, STAT=istat)

      END SUBROUTINE principal_value

C         ***********************************************
C             Set up of relative mesh and weights
C         ***********************************************

      SUBROUTINE rel_mesh
      USE mesh_variables
      IMPLICIT NONE
      INTEGER i
      DOUBLE PRECISION pih,u,s,xx,c,h_max
      PARAMETER (c=0.75)
      DIMENSION u(n_rel), s(n_rel)

      pih=ACOS(-1.D0)/2.D0
      CALL gausslegendret (0.D0,1.d0,n_rel,u,s)
      DO i=1,n_rel
         xx=pih*u(i)
         ra(i)=DTAN(xx)*c
         wra(i)=pih*c/DCOS(xx)**2*s(i)
      ENDDO

      END SUBROUTINE rel_mesh

C     *********************************************************
C            Routines to do mtx inversion, from Numerical
C            Recepies, Teukolsky et al. Routines included
C            below are MATINV, LUDCMP and LUBKSB. See chap 2
C            of Numerical Recepies for further details
C            Recoded in FORTRAN 90 by M. Hjorth-Jensen
C     *********************************************************

      SUBROUTINE matinv(a,n)
      IMPLICIT REAL*8(A-H,O-Z)
      DIMENSION a(n,n)
      INTEGER istat
      DOUBLE PRECISION, ALLOCATABLE :: y(:,:)
      INTEGER, ALLOCATABLE :: indx(:)

      ALLOCATE (y( n, n), STAT =istat)
      ALLOCATE ( indx (n), STAT =istat)
      DO i=1,n
         DO j=1,n
            y(i,j)=0.
         ENDDO
         y(i,i)=1.
      ENDDO
      CALL  ludcmp(a,n,indx,d)
      DO j=1,n
         call lubksb(a,n,indx,y(1,j))
      ENDDO
      DO i=1,n
         DO j=1,n
            a(i,j)=y(i,j)
         ENDDO
      ENDDO
      DEALLOCATE ( y, STAT=istat)
      DEALLOCATE ( indx, STAT=istat)

      END SUBROUTINE matinv
 

      SUBROUTINE LUDCMP(A,N,INDX,D)
      IMPLICIT REAL*8(A-H,O-Z)
      PARAMETER (TINY=1.0E-20)
      DIMENSION A(N,N),INDX(N)
      INTEGER istat
      DOUBLE PRECISION, ALLOCATABLE :: vv(:)

      ALLOCATE ( vv(n), STAT = istat)
      D=1.
      DO I=1,N
         AAMAX=0.
         DO J=1,N
            IF (ABS(A(I,J)) > AAMAX) AAMAX=ABS(A(I,J))
         ENDDO
         IF (AAMAX == 0.) PAUSE 'Singular matrix.'
         VV(I)=1./AAMAX
      ENDDO
      DO J=1,N
         IF (J > 1) THEN
            DO I=1,J-1
               SUM=A(I,J)
               IF (I > 1)THEN
                  DO K=1,I-1
                     SUM=SUM-A(I,K)*A(K,J)
                  ENDDO
                  A(I,J)=SUM
               ENDIF
            ENDDO
         ENDIF
         AAMAX=0.
         DO I=J,N
            SUM=A(I,J)
            IF (J > 1)THEN
               DO K=1,J-1
                  SUM=SUM-A(I,K)*A(K,J)
               ENDDO
               A(I,J)=SUM
            ENDIF
            DUM=VV(I)*ABS(SUM)
            IF (DUM >= AAMAX) THEN
               IMAX=I
               AAMAX=DUM
            ENDIF
         ENDDO
         IF (J /= IMAX)THEN
            DO K=1,N
               DUM=A(IMAX,K)
               A(IMAX,K)=A(J,K)
               A(J,K)=DUM
            ENDDO
            D=-D
            VV(IMAX)=VV(J)
         ENDIF
         INDX(J)=IMAX
         IF(J /= N)THEN
            IF(A(J,J) == 0.) A(J,J)=TINY
            DUM=1./A(J,J)
            DO I=J+1,N
               A(I,J)=A(I,J)*DUM
            ENDDO
         ENDIF
      ENDDO
      IF(A(N,N) == 0.)  A(N,N)=TINY
      DEALLOCATE ( vv, STAT = istat)

      END SUBROUTINE LUDCMP
 
      SUBROUTINE LUBKSB(A,N,INDX,B)
      implicit real*8(a-h,o-z)
      DIMENSION A(N,N),INDX(N),B(N)
      II=0
      DO I=1,N
         LL=INDX(I)
         SUM=B(LL)
         B(LL)=B(I)
         IF (II /= 0)THEN
            DO J=II,I-1
               SUM=SUM-A(I,J)*B(J)
            ENDDO
         ELSE IF (SUM /= 0.) THEN
            II=I
         ENDIF
         B(I)=SUM
      ENDDO
      DO I=N,1,-1
         SUM=B(I)
         IF (I < N)THEN
            DO J=I+1,N
               SUM=SUM-A(I,J)*B(J)
            ENDDO
         ENDIF
         B(I)=SUM/A(I,I)
      ENDDO

      END SUBROUTINE lubksb
\end{lstlisting}
\end{comment}
The parameters of the box potential are chosen to fit a
potential with a bound state at zero energy. What does this mean
for your $T$-matrix with this potential when
$k\rightarrow 0$?






\end{doconceexercise}
% --- end exercise ---














\end{document}
